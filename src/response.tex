\documentclass[11pt]{article}

\usepackage{amsmath,latexsym,amssymb} 
\usepackage{fullpage,ifthen}
\usepackage{color,soul,colordvi}
\usepackage{mdwlist,paralist}

%\renewcommand{\baselinestretch}{1.02}

%\parskip=\smallskipamount

\sloppy

%%%%%%%%%%%%%%%%%%%%%%%%%%%%%%%%%%%%%%%%%%%%%%%%%%%%%%%%%%%%%%%%%%%%%

\newcommand{\abs}[1]{\left| #1 \right|}
\newcommand{\ceil}[1]{\left\lceil {#1} \right\rceil}
\newcommand{\floor}[1]{\left\lfloor {#1} \right\rfloor}
\newcommand{\paren}[1]{\left( #1 \right)}
\newcommand{\set}[1]{\left\{ #1 \right\}}

\newcommand{\eps}{\varepsilon}

\newcommand{\calC}{\mathcal{C}}
\newcommand{\sigmad}{\sigma_{\$}}

\newcommand{\ans}[1]{\begin{quotation}{\noindent \sl {#1}}\end{quotation}}


\newcommand{\dror}[1]{\sethlcolor{yellow}\hl{Dror: #1}}
\newcommand{\ben}[1]{\sethlcolor{green}\hl{Ben: #1}}

%%%%%%%%%%%%%%%%%%%%%%%%%%%%%%%%%%%%%%%%%%%%%%%%%%%%%%%%%%%%%%%%%%%%%

\begin{document}

\title{\textbf{Service Chain Placement in SDN} \\
  {\Large Response to Reviewers: Submission DA10028}}

\author{Gilad Kutiel \and Dror Rawitz}

\maketitle

%%%%%%%%%%%%%%%%%%%%%%%%%%%%%%%%%%%%%%%%%%%%%%%%%%%%%%%%%%%%%%%%%%%%%

We thank the referee for the time that was spent on reviewing our
paper and for their thoughtful comments.  The paper was revised
according to these comments.

%%%%%%%%%%%%%%%%%%%%%%%%%%%%%%%%%%%%%%%%%%%%%%%%%%%%%%%%%%%%%%%%%%%%%

\medskip

\section*{Response to Review \#1}

\begin{enumerate}

\item \textbf{Comment}: In Sec.~5.1 for the randomized algorithm it is
  claimed in p.11 row 29, that ``... each edge $e$ in $E$ is found on
  a simple path between two neighborly nodes $v_e$ and $v_{e'}$, in
  which all internal vertices are in $V \ N$''.  $N$ denotes the set
  of neighborly nodes. Here it is not clear for me, how this is ensured
  if, e.g, $G$ is a tree and $N$ is the set of internal nodes with
  degree greater than two.  It needs more explanation. The approach
  described in Sec 5.2 eliminates this problem.

\item[] \textbf{Response}: Before Section 5.1 we observe, that
  w.l.o.g., we may focus on vertices that are located on a path from
  $s$ to $t$.  It follows that each such vertex has degree at least
  $2$.
%
  In Section~5.1 we now consider edges that are located between $s$ or
  $t$ and a neighborly vertex.
  
\item \textbf{Comment}: In proof of Lemma 1, p.10 row 16: $\leq \eps
  c_{\max} / n \cdot \sum_\alpha (c(\alpha,f(\alpha))\cdot \eps
  c_{max} / n + 1)$ $\longrightarrow$ $\leq \eps c_{\max} / n \cdot
  \sum_\alpha (c(\alpha,f(\alpha))\cdot n / (\eps c_{max}) + 1)$

\item[] \textbf{Response}: Corrected.

\item \textbf{Comment}: In p.~15 row 24: ``... for $v_1 \neq v_2$
  ...'' I suppose, it must be ``... for $v_1 \neq v_2$ or $v_1 = v_2 =
  t$ ...''

\item[] \textbf{Response}: Corrected.

\item \textbf{Comment}: In p.~15 row 34: ``$\beta_i \preceq \alpha_i$
  or $\beta_i = \zeta_i$''.  What does then $\text{COST}^i(\zeta_i \to
  \alpha_i)$ mean in row 26?

\item[] \textbf{Response}: First, we changed $i$ to $j$, since $i$ is
  used for the layer number.

  If $\beta_j = \zeta_j$, then $\textsc{cost}^j_{v_j}(\zeta_j \to
  \alpha_j) = 0$.  We added a footnote about this issue.

\item \textbf{Comment}: In p.~15 after row 37, to my opinion, an
  additional item should be inserted: $v_1 \neq v_2$ or $v_1 = v_2 =
  t$.

\item[] \textbf{Response}: First, this is handled by the response to
  comment 3.  Also, $v_1$ and $v_2$ are not variables over which the
  minimum is taken.

\end{enumerate}



%%%%%%%%%%%%%%%%%%%%%%%%%%%%%%%%%%%%%%%%%%%%%%%%%%%%%%%%%%%%%%%%%%%%%

\newpage

\section*{Response to Review \#2}


We corrected all typos in the list.


\subsection*{Additional references/explanations}

\begin{enumerate}
  
\item \textbf{Comment}: Please expand the explanations for Figure 1,
  the introductory example. For a first-time reader, it is not
  intuitive.

\item[] \textbf{Response}: Done.

\item \textbf{Comment}: Please expand the explanation at the end of
  Theorem 2 a bit.

\item[] \textbf{Response}: Done.
  
\item \textbf{Comment}: Please provide a reference to the FPTAS for
  the MINIMUM KNAPSACK problem.

\item[] \textbf{Response}: Done.

\item \textbf{Comment}: Please expand the explanation at the end of
  Theorem 5 a bit and/or provide a citation.

\item[] \textbf{Response}: The proof was extended.

\item \textbf{Comment}: Please provide a reference for the block-cut
  tree and expand the explanation a bit.

\item[] \textbf{Response}: Done.
  
\item \textbf{Comment}: Please expand the proof of Theorem 4, it is a
  bit short at the moment (maybe it is obvious and I don't see it?)

\item[] \textbf{Response}: Done.
  
\item \textbf{Comment}: Please define internal vertex disjoint on page
  12.

\item[] \textbf{Response}: The definition was added in a footnote.

\item \textbf{Comment}: Please expand the explanation at the end of
  Theorem 9, I don't see it right now.

\item[] \textbf{Response}: The explanations before and after Theorem 9
  were extended.
  
\item \textbf{Comment}: reference 5 has been extended at TOCS 2018
  (On-Line Path Computation and Function Placement in SDNs)
  https://link.springer.com/article/10.1007/s00224-018-9863-4

\item[] \textbf{Response}: A citation was added.

\item \textbf{Comment}: reference 8 has been extended for the
  unordered case (LATIN 2018 (Walking Through Waypoints):
  https://link.springer.com/chapter/10.1007\%2F978-3-319-77404-6\_4),
  and the ordered case (CCR 2018 (Charting the Algorithmic Complexity
  of Waypoint Routing):
  https://dl.acm.org/citation.cfm?doid=3211852.3211859, and IFIP
  Networking 2018 (Waypoint Routing in Special Networks):
  https://files.ifi.uzh.ch/stiller/IFIP\%20Networking\%202018-Proceedings.pdf).
  
\item[] \textsc{Response}: The paper are mentioned in the intro.

\item \textbf{Comment}: Lastly, reference 15 has been extended at IFIP
  Networking 2018 as well (see
  https://files.ifi.uzh.ch/stiller/IFIP\%20Networking\%202018-Proceedings.pdf). The
  same authors also presented some prior work in Computer Networks
  (https://www.sciencedirect.com/science/article/abs/pii/S138912861500359X?via\%3Dihub)
  which should be briefly discussed

\item[] \textsc{Response}: The paper are mentioned in the intro.

\item \textbf{Comment}: Please include a small conclusion with open
  questions at the end. For example, how about walks instead of paths?
  How about multiple concurrent requests?

\item[] \text{Response}: Added.

\end{enumerate}

\end{document}

