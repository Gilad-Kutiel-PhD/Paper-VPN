%\lipsum[1-2]

We study the allocation problem of a \emph{service chain} in a
software-defined network that supports network function
virtualization.
%
Given a network that contains servers with limited processing power
and of links with limited bandwidth, a service chain is a sequence of
virtual network functions (VNFs) that service a certain flow request
in the network.
%
The allocation of a service chain consists of routing and VNF
placement.  That is, each VNF from the seqeunce is placed in a server
along a path, and it is feasible if each server can handle the VNFs
that are assigned to it, and if each link on the path can carry the
flow that is assigned to it.
%
A request for service is composed of a source and a destination in the
network, an upper bound on the total latency, and a specification of
all service chains that are considered valid for this request.
%
Each pair of server and VNF are associated with a cost for placing the
network function in the server, and given a request, the goal is to
find a valid service chain of minimum total cost or to identify that
a valid service chain does not exist.

We show that even the feasibility problem is NP-hard in general
di-graphs.  Hence we first focus on directed acyclic graphs (DAGs).
We show that the problem is still NP-hard in DAGs and present an
FPTAS.
%
Based on out PTAS, we also provide a randomized algorithm for
instances in which the service chain passes through a bounded number
of servers whose degree is larger than two.

?????

Fault tolerant placement: two service chains?

FPT algorithm where the number of servers whose degree is larger than
two is a parameter?
