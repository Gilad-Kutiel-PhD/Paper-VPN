\nonstopmode
%\documentclass[11pt]{article}
\documentclass[runningheads]{llncs}

\usepackage{amsmath,amssymb,mathtools}
%\usepackage{amsthm}
%\usepackage{fullpage}
\usepackage{xspace}
\usepackage{color,soul}
\usepackage{paralist}

% \usepackage[linesnumbered, ruled]{algorithm2e}

\usepackage[final]{microtype}
\usepackage[final]{hyperref}
\usepackage{subcaption}

% DEBUG
%\usepackage{lipsum}
%\usepackage{todonotes}
%\usepackage{lineno}
%\linenumbers

% EXTRA
%\usepackage{authblk}

\usepackage{tikz}
\usetikzlibrary{
	arrows,
	arrows.meta,
	calc,
	patterns,
	positioning,
	shapes,
	decorations.pathmorphing,
	decorations.markings,
}

% \usetikzlibrary{graphs}
\usetikzlibrary{graphdrawing}
\usegdlibrary{trees,force,layered}

%DEFAULT STYLE
\tikzset{default style/.style={
	>=Stealth, 
	on grid, 
	auto, 
	thick,
}}

% NODE
\tikzset{default node/.style={
	draw, 
	circle,
	inner sep=0mm,
	minimum size=5mm,
	very thick,
	font=\small,
	black!70,
}}

\tikzset{clear node/.style={
	draw=none, 
	inner sep=0mm,
	minimum size=0mm,
}}

% LABELS
\tikzset{label/.style={
	draw=none,
	sloped,
}}

\tikzset{label above/.style={
	label,
	midway,
	above=-.2mm,
}}

\tikzset{label below/.style={
	label,
	midway,
	below=-.2mm,
}}


\def\VPN{Service Chain Placement in SDN}
\def\R{\mathbb{R}}
\def\N{\mathbb{N}}


\sloppy


%%%%%%%%%%%%%%%%%%%%%%%%%%%%%%%%%%%%%%%%%%%%%%%%%%%%%%%%%%%%%%%%%%%%%

%% handy commands

\newcommand{\eqdf}{\stackrel{\scriptscriptstyle \triangle}{=}}
\newcommand{\argmin}{\ensuremath{\mbox{argmin}}}
\newcommand{\argmax}{\ensuremath{\mbox{argmax}}}
\newcommand{\set}[1]{\left\{ #1 \right\}}
\newcommand{\paren}[1]{\left( #1 \right)}
\newcommand{\bit}{\set{0,1}}
\newcommand{\inv}[1]{\frac{1}{#1}}
\newcommand{\abs}[1]{\left| #1 \right|}
\newcommand{\ceil}[1]{\left\lceil {#1} \right\rceil}
\newcommand{\floor}[1]{\left\lfloor {#1} \right\rfloor}
\newcommand{\half}{\frac{1}{2}}
\newcommand{\threehalves}{\frac{3}{2}}

\newcommand{\naturals}{\mathbb{N}}
\newcommand{\rationals}{\mathbb{Q}}
\newcommand{\reals}{\mathbb{R}}
\newcommand{\integers}{\mathbb{Z}}

\newcommand{\eps}{\varepsilon}

%%%%%%%%%%%%%%%%%%%%%%%%%%%%%%%%%%%%%%%%%%%%%%%%%%%%%%%%%%%%%%%%%%%%%

\newcommand{\scp}{\textsc{SCP}\xspace}
\newcommand{\scplong}{\textsc{Service Chain Placement}\xspace}

\newcommand{\calE}{\mathcal{E}}
\newcommand{\calG}{\mathcal{G}}
\newcommand{\calV}{\mathcal{V}}

\newcommand{\dror}[1]{\sethlcolor{yellow}\hl{Dror: #1}}


%%%%%%%%%%%%%%%%%%%%%%%%%%%%%%%%%%%%%%%%%%%%%%%%%%%%%%%%%%%%%%%%%%%%%

\title{\bf Service Chain Placement in SDNs%
\thanks{Research supported in part by Network Programming (Neptune)
  Consortium, Israel.}
}

\titlerunning{Service Chain Placement in SDNs}


%\author[1]{Gilad Kutiel%
%\thanks{E-mail: \texttt{gkutiel@cs.technion.ac.il}}}

%\author[2]{Dror Rawitz%
%\thanks{Partially supported by the Israel Science Foundation
%  (grant no.~497/14).  E-mail: \texttt{dror.rawitz@biu.ac.il}}}

%\affil[1]{Department of Computer Science, Technion, Haifa, Israel}
%\affil[2]{Faculty of Engineering, Bar Ilan University, Ramt-Gan, Israel}

\author{%
Gilad Kutiel
\and
Dror Rawitz%
\thanks{Partially supported by the Israel Science Foundation
  (grant no.~497/14).  }
}

\institute{%
Department of Computer Science, Technion, Haifa 32000, Israel. \\
\email{gkutiel@cs.technion.ac.il}
\and
Faculty of Engineering, Bar Ilan University, Ramt-Gan 52900, Israel.\\
\email{dror.rawitz@biu.ac.il}
}

%%%%%%%%%%%%%%%%%%%%%%%%%%%%%%%%%%%%%%%%%%%%%%%%%%%%%%%%%%%%%%%%%%%%%

\begin{document}
\maketitle

\begin{abstract}
We study the allocation problem of a \emph{service chain} in a
software-defined network that supports network function
virtualization.
%
Given a network that contains servers with limited processing power
and of links with limited bandwidth, a service chain is a sequence of
virtual network functions (VNFs) that service a certain flow request
in the network.
%
The allocation of a service chain consists of routing and VNF
placement.  That is, each VNF from the seqeunce is placed in a server
along a path, and it is feasible if each server can handle the VNFs
that are assigned to it, and if each link on the path can carry the
flow that is assigned to it.
%
A request for service is composed of a source and a destination in the
network, an upper bound on the total latency, and a specification of
all service chains that are considered valid for this request.
%
Each pair of server and VNF are associated with a cost for placing the
network function in the server, and given a request, the goal is to
find a valid service chain of minimum total cost or to identify that
a valid service chain does not exist.

We show that even the feasibility problem is NP-hard in general
di-graphs.  Hence we first focus on directed acyclic graphs (DAGs).
We show that the problem is still NP-hard in DAGs and present an
FPTAS.
%
Based on out PTAS, we also provide a randomized algorithm for
instances in which the service chain passes through a bounded number
of servers whose degree is larger than two.

?????

Fault tolerant placement: two service chains?

FPT algorithm where the number of servers whose degree is larger than
two is a parameter?
\end{abstract}

%%%%%%%%%%%%%%%%%%%%%%%%%%%%%%%%%%%%%%%%%%%%%%%%%%%%%%%%%%%%%%%%%%%%%

\section{Introduction}


Computer communication networks are in constant need of expansion to
cope with the ever growing traffic.  As networks grow, management and
maintenance become more and more complicated.
%
Current developments aim to improve the utilization of network
resources include the detachment of network applications from network
infrastructure and the transition from network planning to network
programming.

One aspect of network programming is to manage resources from a
central point of view, namely to make decisions based on availability,
network status, required quality of service, and the identity of the
client.  Hence, a central issue is a central agent that is able to
received reports from network components and client requests, and as a
result can alter the allocation of resources in the networks.  This
approach is called Software Defined Networking (SDN), where there is a
separation between routing and management (control plane) and the
underlying routers and switches that forward traffic (data plane) (see
Kreutz et al.~\cite{KRVRAU15}).

A complimentary approach is Network Function Virtualization
(NFV)~\cite{NFV12}.  Instead of relying on special purpose machines,
network applications become virtual network functions (VNF) that are
executed on generic machines and can be placed in various locations in
the network.  Virtualization increases the flexibility of resource
allocation and thus the utilization of the network resources.
%
Internet Service Providers (ISPs) that provide services to clients
benefit from NFV, since it helps to better utilize their physical
network.  In addition, when network services are virtualized, an ISP
may support \emph{service chaining}~\cite{ServiceChaining15}, namely a
compound service that consists of a sequence of VNFs.  Furthermore,
one may offer a compound service a service that may be obtained using
one of several sequences of VNFs.

In this paper we study the allocation problem of a compund request for
a service chain in a software-defined network that supports NFV.
Given a (physical) network that contains servers with limited
processing power and of links with limited bandwidth, a service chain
is a sequence of virtual network functions (VNFs) that service a
certain flow request in the network.
%
The allocation of a service chain consists of routing and VNF
placement.  That is, each VNF from the seqeunce is placed in a server
along a path, and it is feasible if each server can handle the VNFs
that are assigned to it, and if each link on the path can carry the
flow that is assigned to it.
%
A request for service is composed of a source and a destination in the
network, an upper bound on the total latency, and a specification of
all service chains that are considered valid for this request.
%
Each pair of server and VNF are associated with a cost for placing the
network function in the server.  This cost measures compatibility of a
VNF to a server (infinite costs means ``incompatible'').  Given a
request, the goal is to find a valid service chain of minimum total
cost or to identify that a valid service chain does not exist.


\dror{Explain one shot approach}

\paragraph*{\bf Related work.}
%

Cohen et al.~\cite{CLNR15} considered VNF placement.  In their model
the input is an undirected graph with a metric distance function on
the edges.  clients are located in various nodes, and each client is
interested in a subset of VNFs.  For each node $v$ and VNF $\alpha$,
there is a setup cost associated with placing a copy of $\alpha$ at
$v$ (multiple copies are allowed), and there is a load that is induced
on $v$ for placing a copy of $\alpha$.  Furthermore, each node has a
limited capacity.  Also, each copy of a VNF can handle a limited
amount of clients A solution is the assignment of each client to a
subset of nodes, each corresponding to one of its required VNFs.  The
cost of a solution is the total setup costs plus the sum of distances
between the clients and the node from which they get service.
%
Cohen et al.~\cite{CLNR15} gave bi-criteria approximation algorithms
for various versions of the problem, namely algorithms that compute
constant factor approximations that violate the load constraints by a
constant factor.
%
It is important to note that in this problem does not consider
routing, and also it assumes unordered VNF subsets.

Even, Medina, and Patt-Shamir~\cite{EMP16} studied online path
computation and VNF placement.  They considered service chain requests
that arrive in an online manner.  Each request is a a flow with a
high-level specification of routing and VNF requirements.  Each VNF
can be performed by a specified subset of servers in the system.  Upon
arrival of a request, the algorithm either rejects the request or
accepts it and with a specific routing and VNF assignment, under given
server capacity constraints.  Each request has a benefit that is
gained if it is accepted, and the goal is to maximize the total
benefit.
%
Even, Medina, and Patt-Shamir~\cite{EMP16} proposed an algorithm that
copes with requests with unknown duration without preemption by using
a third response, which is refer to as “stand by”, whose competitive
ratio is $O(\log (knb_{\max})$, where $n$ is the number of nodes,
$b_{\max}$ is an upper bound on a request benefit per time unit, and
$k$ is the maximum number of VNFs of any service chain.  This
performance guarantee holds for the case where the processing of any
processing request in any possible service chain is never more than
$O(1/(k \log (nk)))$ fraction of the capacity of any network
component.

Even, Rost, and Schmid~\cite{ERS16}

\dror{finish}

\paragraph*{\bf Our results.}
%
We show that even the feasibility problem is NP-hard in general
di-graphs.  Hence we first focus on directed acyclic graphs (DAGs).
We show that the problem is still NP-hard in DAGs and present an
FPTAS.
%
Based on out PTAS, we also provide a randomized algorithm for
instances in which the service chain passes through a bounded number
of servers whose degree is larger than two.

?????

\bigskip



%%%%%%%%%%%%%%%%%%%%%%%%%%%%%%%%%%%%%%%%%%%%%%%%%%%%%%%%%%%%%%%%%%%%%

\section{Preliminaries}


\paragraph*{\bf Model.}
%
The \scplong (\scp) input is composed of three components, a physical
network, a virtual specification, and placement costs:
\begin{description}
\item[Physical network:]
%
  The physical network is a digraph $G = (V,E)$.  Each node $v \in V$
  has a non-negative processing capacity $p(v)$, and each directed
  edge $e \in A$ has a non-negative bandwidth cap $b(e)$.

\item[Virtual specification:]
%
  The description of a request for a service chain consists of a
  physical source $s \in V$, a physical destination $t \in V$, and a
  directed acyclic graph (DAG) $\calG = (\calV,\calE)$.
%
  Without loss of generality, we assume that $p(s) = p(t) = 0$.  
%
  The DAG $\calG$ has a source node $\sigma \in \calV$ and a
  destination $\tau \in \calV$ Each node $\alpha \in \calV$ represent
  a VNF that has a processing demand $p(\alpha)$.  Without loss of
  generality, we assume that $p(\sigma) = p(\tau) = 0$.  Each edge
  $\eps \in \calE$ has a bandwidth demand $b(\eps)$.
  
\item[Placement costs:]
%
  There is a non-negative cost $c(\alpha,v)$ for placing the VNF
  $\alpha$ in $v$.  We assume that $c(\sigma,s) = 0$ and $c(\sigma,v)
  = \infty$, for every $v \neq s$.  Similarly, we assume that
  $c(\tau,t) = 0$ and $c(\tau,v) = \infty$, for every $v \neq t$.
\end{description}

A solution consists of the following:
\begin{description}
\item[Virtual path:] A path from $\sigma$ to $\tau$ in $\calG$, namely
  a sequence of vertices $\alpha_0,\ldots,\alpha_\ell$, where
  $\alpha_0 = \sigma$, $\alpha_\ell = \tau$, and
  $(\alpha_j,\alpha_{j+1}) \in \calE$, for every $j \in
  \set{0,\ldots,\ell-1}$.

\item[Physical path:] A simple path from $s$ to $t$ in $G$, namely a
  sequence of node $v_0,\ldots,v_k$, where $v_0 = s$, $v_k = t$, and
  $(v_i,v_{i+1}) \in E$, for every $i \in \set{0,\ldots,k-1}$.
  
\item[Placement:] A function $f$ that maps the virtual path to the
  physical path.  Formally, a placement is a function $f:
  \set{\alpha_0,\ldots,\alpha_\ell} \to \set{v_0,\ldots,v_k}$
  such that
\begin{enumerate}
\item If $f(\alpha_j) = v_i$ and $f(\alpha_{j+1}) = v_{i'}$, then $i
  \leq i'$.
\item $b(\alpha_j,\alpha_{j+1}) \leq b(v_i,v_{i+1})$, for every
  $(v_i,v_{i+1}) \in \bar{f}(\alpha_j,\alpha_{j+1})$, where
  $\bar{f}(\alpha_j,\alpha_{j+1})$ to be the set of physical edges
  that correspond to it, i.e.,
\[
\bar{f}(\alpha_j,\alpha_{j+1}) \eqdf \set{(v_i,v_{i+1}) : i' \leq i < i''}
~,
\]
  where $f(\alpha_j) = v_{i'}$ and $f(\alpha_{j+1}) = v_{i''}$.

\item $\sum_{\alpha \in f^{-1}(v)} p(\alpha) \leq p(v)$, where
  $f^{-1}(v) \eqdf \set{\alpha : f(\alpha) = v}$.
 
\end{enumerate}
\end{description}

An example of an \scp instance and a solution is given in
Figure~\ref{fig:solution}.

\begin{figure}
\centering
\scalebox{1}{
\begin{tikzpicture}[every node/.style={default node}, very thick]

\begin{scope}[random seed=1, spring layout, node distance=1.8cm]
\node(a) {a};
\node(b) {b};
\node(c) {c};
\node(d) {d};
\node(e) {e};
\node(f) {f};
\node(g) {g};
\node(h) {h};
\node(i) {i};
\node(j) {j};

\graph{
	(a) -> (b);
	(b) -> {(d), (e)};
	(d) -> {(f), (h)};
	(f) -> {(g), (h), (c)};
	(h) -> {(g), (i), (e)};
	{[edges={blue, dashed}]
		(a) -> (c) -> (d) -> (e) -> (i) -> (g) -> (j);
	}
};

\node(pa) {a'};
\node(pb) {b'};
\node(pc) {c'};
\node(pd) {d'};
\node(pe) {e'};
\node(pf) {f'};
\node(pg) {g'};
\node(ph) {h'};

\graph{
	(pa) -> (pb);
	(pb) -> {(pd), (pe)};
	(pd) -> {(pf), (ph)};
	(pf) -> {(pg), (ph), (pc)};
	(ph) -> {(pg), (pe)};
	(pe) -> (ph);
};
\end{scope}

\begin{scope}[dotted, orange, -]
\draw
(a.north east) to[out=135, in=45] node(j1)[midway, clear node] {}
(c.north west) to[out=225, in=135]
(d.south west) to[out=-45, in=-45]
(a.north east)
;

\draw
(e.east) to[out=-90, in=-90] node(j2)[midway, clear node] {}
(i.west) to[out=90, in=90]
(e.east)
;

\draw
(g.east) to[out=-90, in=-90] node(j3)[midway, clear node] {}
(j.west) to[out=90, in=90]
(g.east)
;

\draw[->] (j1) to[bend left] (pa);
\draw[->] (j2) to[bend right] (pe);
\draw[->] (j3) to[bend right] (pg);
\end{scope}
	

\node[clear node] at(-5,0) {Virtual Graph};
\node[clear node] at(5, -2){Physical Graph};

\end{tikzpicture}
}
\caption{???}
\label{fig:solution}
\end{figure}

The cost of a placement $f$ is:
\[
\sum_j c(\alpha_j,f(\alpha_j))
~.
\]
In \scp the goal is to find a feasible placement of a virtual path
into the physical DAG that minimizes the cost.

%%%%%

We also consider an extended version of \scp in which each physical
link causes a delay and there is an upper bound $T$ on the total
delay.  More formally, each pair of a virtual edge $\eps$ and physical
edge $e$ is associated with a latency $t(\eps,e)$.
%
Given a placement $f: \set{\alpha_0,\ldots,\alpha_\ell} \to
\set{v_0,\ldots,v_k}$, the total latency of the solution is given by
\[
T(f) = \sum_j \sum_{e \in \bar{f}(\eps)} t(\eps,e)
~.
\]
In this case there is an additional constraint that $T(f) \leq T$.


%%%%%

\paragraph*{\bf Definitions and notation.}
%
Define $n_p \eqdf \abs{V}$ and $m_p \eqdf \abs{E}$.  Similarly, define
$n_v \eqdf \abs{\calV}$ and $m_v \eqdf \abs{\calE}$.

Given a virtual DAG $\calG$, we assume the existence of a topological
sorting of the vertices and write $\alpha \prec_v \beta$ if $\alpha$
precedes $\beta$ in this ordering.  If the physical network is a DAG,
then we make a similar assumption and write $v \prec_p u$ if $v$
precedes $u$ in this ordering.




%%%%%%%%%%%%%%%%%%%%%%%%%%%%%%%%%%%%%%%%%%%%%%%%%%%%%%%%%%%%%%%%%%%%%

\section{Hardness}

In this section we present two hardness results.  First, we show that
even the feasibility question of \scp is NP-hard, and therefore no
approximation algorithm exists for \scp.  In addition, we show that
\scp is NP-hard even if $\abs{V \setminus \set{s,t}} = 1$.  We also
show that \scp is NP-hard even if both the physical network and the
virtual DAG are simple paths.

We show that finding a feasible solution is NP-hard.

\begin{theorem}
Feasibility of \scp is NP-hard
\end{theorem}
\begin{proof}
We use a reduction from \textsc{Hamiltonian Path} that is known to be
NP-hard~\cite{GarJoh79}.
%
Given an instance $H$ of \textsc{Hamiltonian Path} we construct an
instance of \scp.  For the physical network we have that $G$, where
$V(G) = V(H) \cup \set{s,t}$, and $E(G) = E(H) \cup \set{(s,v),(v,t) :
  v \in V(H)}$.  In addition, $p(v) = 1$, for every $v$, and $b(e) =
1$, for every $e$.  The virtual DAG $\calG$ is a path containing $n+2$
virtual functions, $\sigma = \alpha_0, \alpha_1, \ldots, \alpha_n,
\alpha_{n+1} = \tau$, where $p(\alpha_i) = 1$, for every $i \in
\set{1,\ldots,n}$.  Also, $b(\alpha_i,\alpha_{i+1}) = 1$, for every
$i$.

The construction can clearly be computed in polynomial time.
%
An Hamiltonian Path in $G$, induces a \scp solution that follows the
path, i.e., $\alpha_i$ is placed in the $i$th vertex along the path.
%
One the other hand, since all demands and capacities are $1$, no two
VNFs can share a physical node.  Hence a \scp solution induces an
Hamiltonian path.
%
\qed
\end{proof}

Next, we show that \scp is NP-hard even if the physical network is a
DAG.

\begin{theorem}
\scp is NP-hard, even if $\abs{V \setminus \set{s,t}} = 1$.
\end{theorem}
\begin{proof}
The prove the theorem using a reduction from \textsc{Partition}.
Given a \textsc{Partition} instance $\set{a_1, \ldots, a_n}$ we
construct an \scp instance as follows.
%
The physical network $G$ contains three nodes: $s$, $v$, and $t$, and
there are two edges $(s,v)$ and $(v,t)$.  The capacity of $v$ is $p(v)
= \half \sum_i a_i$.  Edge capacities are zero.
%
The virtual DAG is composed of $2n+2$ vertices, namely
\[
\calV
= \set{\sigma,\tau}
  \cup \set{\alpha_1, \ldots, \alpha_n}
  \cup \set{\beta_1, \ldots, \beta_n}
~.
\]
Also,
\[
\calE
= \bigcup_i \set{ \set{\alpha_i,\beta_i} \times \set{\alpha_{i+1},\beta_{i+1}} } 
  \cup
  \set{ (\sigma,\alpha_1), (\sigma,\beta_1), (\alpha_n,t), (\beta_n,t) }
~.
\]
The DAG is shown in Figure~\ref{fig:dag}.
%
The demands are $p(\alpha_i) = a_i$ and $p(\beta_i) = 0$, for every
$i$.  Also, $b(\eps) = 0$, for every $\eps \in \calE$.  The costs are:
$c(\alpha_i,v) = 0$ and $c(\beta_i,v) = a_i$, for every $i$.

The \scp instance can be computed in polynomial time.
%
Also, it is not hard to verify that $a \in \textsc{Partition}$ if and
only if there exists a solution whose cost is $\half \sum_i a_i$.
%
\qed
\end{proof}

\begin{figure}
\centering
\begin{tikzpicture}[every node/.style={default node}]

\begin{scope}[layered layout, grow=right, sibling distance=2cm, level distance=2cm]

\node(v1p) {$\beta_1$};
\node(v1m) {$\alpha_1$};
\node(v2p) {$\beta_2$};
\node(v2m) {$\alpha_2$};
\node[clear node, inner sep=5mm](dotsp) {$\ldots$};
\node[clear node, inner sep=5mm](dotsm) {$\ldots$};
\node(vnp) {$\beta_n$};
\node(vnm) {$\alpha_n$};

\graph{
	s -> {(v1p), (v1m)};
	(v1p) -> {(v2p), (v2m)};
	(v1m) -> {(v2p), (v2m)};
	(v2p) -> {(dotsp), (dotsm)};
	(v2m) -> {(dotsp), (dotsm)};
	(dotsp) -> {(vnp), (vnm)};
	(dotsm) -> {(vnp), (vnm)};
	{(vnp), (vnm)} -> t;
};

\end{scope}

\end{tikzpicture}

\caption{???}
\label{fig:dag}
\end{figure}

\dror{Figure 2: source and destination should be fixed}

\begin{theorem}
\scp is NP-hard, even if both the physical network and the virtual DAG
are paths.
\end{theorem}
\begin{proof}
 
Given a partition instance $a_1, \ldots, a_n$, we construct a virtual path
$\alpha_1, \ldots, \alpha_n, \tau$ and a physical path 
$v_1^-, v_1^+, \ldots, v_n^-, v_n^+, t$.
We set $p \equiv 1$, $b \equiv 1$, 
$c(\alpha_i, v_i^-) = 0$, $c(\alpha_i, v_i^+) = a_i$, $c(\tau, t) = 0$
and for any $v \notin \{v_i^-,v_i^+\}$ we set $c(\alpha_i, v) = \infty$.
We also set $l(\alpha_i\alpha_{i+1}, v_i^-v_i^+) = a_i$, and set $l = 0$
otherwise.
Let $s = 1/2\sum_i^n a_i$
One can verify that there is embedding with cost $s$ and latency $s$ 
if and only if $a_1, \ldots, a_n$ can be partitioned.
\qed
\end{proof}
Figure~\ref{fig:reduction2} depicts the above reduction.
\begin{figure}[ht]
\centering
\scalebox{.9}{
\begin{tikzpicture}[every node/.style={default node}]
%\node[draw=none] at(3, 1) {Virtual Path};

\node(sigma) at (-2,0) {$\sigma$};
\node(alpha1) at(0,0) {$\alpha_1$};
\node(alpha2) at(2,0) {$\alpha_2$};
\node[draw=none] at(6,0) {$\cdots$};
\node(alphan) at(10,0) {$\alpha_n$};
\node(tau) at(12,0) {$\tau$};

\draw (sigma) -- (alpha1);
\draw (alpha1) -- (alpha2);
\draw (alpha2) -- +(2, 0);
\draw ($(alphan) + (-2,0)$) -- (alphan);
\draw (alphan) -- (tau);

\node(s) at (-2,-4) {$s$};
\node(v1-) at(0, -2) {$v_1^-$};
\node(v1+) at(0, -4) {$v_1^+$};
\node(v2-) at(2, -2) {$v_2^-$};
\node(v2+) at(2, -4) {$v_2^+$};

\node[draw=none] at(6,-2) {$\cdots$};
\node[draw=none] at(6,-4) {$\cdots$};

\node(vn-) at(10, -2) {$v_n^-$};
\node(vn+) at(10, -4) {$v_n^+$};
\node(t) at(12, -2) {$t$};

\draw (s) -- (v1-) node[label] {$0$};
\draw (v1-) -- (v1+) node[label] {$a_1$};
\draw (v1+) -- (v2-) node[label] {$0$};
\draw (v2-) -- (v2+) node[label] {$a_2$};
\draw (v2+) -- +(2, 2)node[label] {$0$};

\draw ($(vn-) + (-2,-2)$) -- (vn-)node[label] {$0$};;
\draw (vn-) -- (vn+)node[label] {$a_n$};
\draw (vn+) -- (t)node[label] {$0$};;

\begin{scope}[every node/.style={label}, dotted, purple, ->]
\draw (sigma) to[bend right] node {$0$} (s);

\draw (alpha1) to[bend left] node {$0$} (v1-);
\draw (alpha1) to[bend right] node {$a_1$} (v1+);
\draw (alpha2) to[bend left] node {$0$} (v2-);
\draw (alpha2) to[bend right] node {$a_2$} (v2+);
\draw (alphan) to[bend left] node {$0$} (vn-);
\draw (alphan) to[bend right] node {$a_n$} (vn+);

\draw (tau) to[bend left] node {$0$} (t);
\end{scope}

\end{tikzpicture}

}
\caption[]{
\label{fig:reduction2}
A reduction from a partition instance $a_1, \ldots, a_n$.
Each $\alpha_i$ can be placed either in $v_i^-$ or in $v_i^+$.
In the former case this placement incure no cost but additional latency of $a_i$,
in the later case there will be additional $a_i$ cost with zero latency.
}
\end{figure}


%%%%%%%%%%%%%%%%%%%%%%%%%%%%%%%%%%%%%%%%%%%%%%%%%%%%%%%%%%%%%%%%%%%%%

\section{Algorithm}
In this section we present an FPTAS algorithm to the \VPN problem.
We are given an instance of the \VPN problem 
\begin{align*}
D_v = (V_v, A_v)		\\
D_p = (V_p, A_p)		\\
(s_v, t_v, s_p, t_p)	\\
(p_d, b_d, p_c, b_c, c)
\end{align*}
For now assume all values are integral and polynomial bounded in the size of the input.

Let $T_{v_p}(u_v, w_v, C)$ denote the minimum processing required to embed
all the virtual vertices, excluding $u_v$, on a $u_vw_v$-path into the physical
vertex $v_p$ with cost constraint $C$.
$T$ can be computed recursively as follow:
\begin{align*}
&T_{v_p}(u_v, w_v, C) = 
\min_{v_vv_v \in A_v} T_{v_p}(u_v, v_v, C - c(w_v, v_p))
\\
&T(u_v, u_v, C) = 0
\\
&T(-, -, 0) = \infty
\end{align*}
 
Let $O(v_v, v_p)$ denote the optimal embedding of a $s_vv_v$-path into a
$s_pv_p$-path.
Let $pi:V_v \to \N$ be any typological ordering of $V_v$. 
We claim that $O$ can be computed recursively as follow:
\begin{align*}
&O(v_v, v_p) = 
\min_{\substack{
u_v : pi(u_v) < pi(v_v), 
\\
u_pv_p \in A_p
}}
O(u_v, u_p)
+
\argmin_{C : T_{v_p}(u_v, v_v, C) \leq p_c(v_p)}
\\
&O(s_v, v_p) = c(s_v, v_p)
\end{align*}


% For a physical vertex $v_p$ we define 
% $c_{v_p}(u_v, v_v):A_v \to \R = c(v_v, v_p)$.
% Given two virtual vertices $s'_v, t'_v$ and a physical vertex $v_p$, 
% one can verify that the minimum cost of embedding any 
% 
% 
% we would like to find an optimal embedding of an $s'_vt'_v$-path into
% $v_p$.
% A solution to this sub-problem can be found as follow:
% First we build the \emph{cost graph} correspond to $s'_v, t'_v$ and $v_p$, 
% the cost graph is a graph that lets us (efficiently) compute the cost of em  
% 
% For a vertex $v$ let $C(v)$ be the set of its children.
% Let $O'(s'_v, t'_v, s'_p, t'_p)$ be the optimal embedding of some virtual
% $C(s'v)t'_v$-path into the physical vertex $t'_p$ such that the bottleneck
% between $s'_p$ and $t'_p$ is at least $b_d(s'_v, t'_v)$.
% Let $c_v \in C(s'_v)$ and observe that once $c_v$ is chosen the arc 
% $(s'_v, c_v)$ determines completely whether a physical path with a feasible
% bottleneck exists, namely, 
% a physical path exists if and only if there is a physical $s'_vt'_v$-path containing only
% arcs with bandwidth at least $b_d(s'_v, c_v)$.
% On the other hand, the set of $c_vt'_v$-paths
% determine completely the cost (and processing-wise feasibility) of the embedding.
% Figure depicts the above observations.
% 
% \begin{figure}[ht]
% \centering
% \begin{tikzpicture}[every node/.style={default node}, very thick, -latex]

\node[clear node] at(-3, 0) {$\dots$};

\node(s) at(-2,0) {$a_v$};

\node(c1) at(1,1) {$c_1$};
\node[clear node] at (2,0) {$\vdots$};
\node(c2) at(1,-1) {$c_d$};

\node(t) at(5, 0) {$t'_v$};

\node[clear node] at(6, 0) {$\dots$};

\draw (s) to[] (c1);
\draw (s) to[]
node[clear node, pos=0.5](c1n){}
(c2);

\begin{scope}[path]
\draw[decorate] (c1.north) to[out=90, in=90] (t.north east);
\draw[decorate] (c1) to[out=45, in=135] (t);
\draw[decorate] (c1.south east) to[out=0, in=135] (t.south west);

\draw[decorate] (c2.north east) to[out=0, in=-135] (t.north west);
\draw[decorate] (c2) to[out=-45, in=-135] (t);
\draw[decorate] (c2.south) to[out=-90,in=-90] (t.south east);
\end{scope}

\draw[group]
(c2.west)		to[out=-90,in=-90, looseness=1.8] node[clear node, pos=0.9](j){}
(t.east)		to[out=90,in=0]
(t.north)		to[out=180,in=45]
(c2.north west)	to[out=-135,in=90]
(c2.west)
;	


\node[clear node] at (-3,-5) {$\dots$};

\node(sp) at(-2, -5) {$s'_p$};
\node(tp) at(5, -5) {$t'_p$};

\node[clear node] at (6,-5) {$\dots$};

\begin{scope}[path]
\draw[decorate]	(sp.north east) to[bend left] 
node[pos=0.5, clear node](p1){}
(tp.north west)
;
\draw[decorate] (sp) to[] 
node[pos=0.3, clear node](p2){}
(tp);
\draw[decorate]	(sp.south east) to[bend right] (tp.south west);
\end{scope}

\draw[group, ->] (j) to[bend left] (tp);

\begin{scope}[blue, -Rays, every to/.style={bend right}]
\draw (c1n) to[] (p1);
\draw (c1n) to[] (p2);
\end{scope}
\end{tikzpicture}

% \caption[]{
% \begin{enumerate*}
% \item Computing $O'(s'_v, t'_v, s'_p, t'_p)$.
% \item The set of $c_dt'_v$-paths determines completely the cost of the embedding
% \item 
% The arc $(s'_v, c_d)$ determines completely the connectivity of $s'_p$ and
% $t'_p$, i.e. physical arcs with bandwidth less than $b_d(s'_v, c_d)$ can not be
% used.
% \end{enumerate*}
% }
% \end{figure}
% 
% Among all the $c_vt'_v$-paths we want to select the one that minimize the
% embedding cost under the processing capacity constraints.
% This can be done... 
% Calculating $O'(s'_v, t'_v, s'_p, t'_p)$ can be done simply by finding the
% optimal path among all children of $s'_v$.
% 
% Let $O(t'_v, t'_p)$ be the optimal embedding of some virtual $s_vt'_v$-path
% into some physical $s_pt'_p$-path then $O(t_v, t_p)$ is the global optimal
% embedding.
% Also $O(t'_v, s_p)$ can be computed efficiently for every $t'_v$ as described
% above, and $O(s_v, t'_v)$ is simply the minimum cost of embedding $s_v$ in any
% physical vertex $s'_p$ such that a physical $s'_pt'_v$-path exists.
% For an arbitrary vertices $t'_v$, and $t'_p$, $O(t'_v, t'_p)$ can be computed
% efficiently using dynamic programming with the following recursion formula:
% $$
% O(t'_v, t'_p) = 
% \min_{\substack{
% s'_v \in P(t'_v)
% \\
% s'_p \in P(t'_p)
% }}
% O(s'_v, s'_p) + O'(s'_v, t'_v, s'_p, t'_p)
% $$ 
% 
%  



%%%%%%%%%%%%%%%%%%%%%%%%%%%%%%%%%%%%%%%%%%%%%%%%%%%%%%%%%%%%%%%%%%%%%

\section{Latency}
We now extend the problem by adding a global latency constraint.
Formally, we are given a latency function $l : A_v \times A_p \to \R$,
and a latency constraint $L$.
The goal is to find a min-cost embedding that also respect the latency
constraint.
Observe that given two physical vertices $s_p$, $t_p$ and a virtual edge $e_v$
a $s_pt_p$-path that minimize the latency can be efficiently found by using any
standard shortest path algorithm where the "distance" function is replaced by
the appropriate latency.
Observe that this can also be done if we restrict ourselves to use only physical
edges with a lower bound on their bandwidth.
We now show that if all values are integral and polynomial bounded, we can
efficiently find a feasible embedding with a cost bounded by $C$ that minimize
the latency.

Let $O(t'_v, t'_p, C)$ be an embedding of some virtual $s_vt'_v$-path 
into a physical $s_pt'_p$-path that has cost not more than $C$ and that
minimizes the latency.



\section{Undirected Graphs}
In this section we consider the problem on undirected (physical) graphs.
Once again we seek to embed a virtual $\sigma\tau$-path 
into a simple physical $st$-path.
Note, that we still assume that the virtual graph is a DAG. 

We extend our algorithm to work also for this case as well.
Recall that the problem on general graphs cannot be approximate at all,
thus we assume in this section that the there exists an optimal embedding using
at most $k$ \emph{heavy} vertices, thus are vertices with degree greater than two.
The modified algorithm runs in time polynomial in the input length and
exponential in $k$ and achieves the same guarantees as before with high
probability.

The modified algorithm acts as follow:
let $K = {v_1, \ldots, v_k}$ be the set of heavy vertices, and let 
$\pi:[k] \to [k]$ be a random permutation.
Let $v_1, v_l \in K$ be two heavy vertices such that $\pi(v_1) < \pi(v_l)$ and
let $w_1, \ldots, w_l$ any light path between them, i.e. a path that does not
contains any other heavy vertex.
We orient the edges from $u$ to $v$, i.e. $(w_i, w_{i+1})$ for every 
$1 \leq i \leq l - 1$.
One can be verify that this can be done efficiently and that after this process
each edge become an arc and that $G$ is a DAG.

Completing the orientation, we use our previous algorithm to find an optimal
embedding.
We repeat this process, each time with a new independent random permutation, and
keep the best embedding so far.

We claim that with high probability, after $O(k!)$ iterations, the above
algorithm finds the optimal embedding, formally:

\begin{theorem}
With probability at least $1 - e^{-x}$ the algorithm finds the optimal embedding
after $xk!$ iterations.
\end{theorem}

\begin{proof}
Consider an optimal embedding and the order it induces on the heavy vertices,
If the random permutation order the heavy vertices in that order we are
guaranteed to find the optimal embedding.
This happens with probability $1/k!$ and the probability to miss the optimal
embedding after $xk!$ iterations is $1 - (1 - 1/k!)^{xk!} \leq 1 - e^{-x}$.
\end{proof}

%%%%%%%%%%%%%%%%%%%%%%%%%%%%%%%%%%%%%%%%%%%%%%%%%%%%%%%%%%%%%%%%%%%%%

\bibliographystyle{abbrv}
\bibliography{chain}

\end{document}
