%\nonstopmode
%\documentclass[11pt]{article}
\documentclass[runningheads]{llncs}

\usepackage{amsmath,amssymb,mathtools}
%\usepackage{amsthm}
%\usepackage{fullpage}
\usepackage{xspace}
\usepackage{color,soul}
\usepackage{paralist}
\usepackage{stmaryrd}

% \usepackage[linesnumbered, ruled]{algorithm2e}

%\usepackage[final]{microtype}
\usepackage[final]{hyperref}
\usepackage{subcaption}

% DEBUG
%\usepackage{lipsum}
%\usepackage{todonotes}
%\usepackage{lineno}
%\linenumbers

% EXTRA
%\usepackage{authblk}

\usepackage{tikz}
\usetikzlibrary{
	arrows,
	arrows.meta,
	calc,
	patterns,
	positioning,
	shapes,
	decorations.pathmorphing,
	decorations.markings,
}

% \usetikzlibrary{graphs}
\usetikzlibrary{graphdrawing}
\usegdlibrary{trees,force,layered}

%DEFAULT STYLE
\tikzset{default style/.style={
	>=Stealth, 
	on grid, 
	auto, 
	thick,
}}

% NODE
\tikzset{default node/.style={
	draw, 
	circle,
	inner sep=0mm,
	minimum size=5mm,
	very thick,
	font=\small,
	black!70,
}}

\tikzset{clear node/.style={
	draw=none, 
	inner sep=0mm,
	minimum size=0mm,
}}

% LABELS
\tikzset{label/.style={
	draw=none,
	sloped,
}}

\tikzset{label above/.style={
	label,
	midway,
	above=-.2mm,
}}

\tikzset{label below/.style={
	label,
	midway,
	below=-.2mm,
}}


\renewcommand{\floatpagefraction}{0.6}

\parskip = \smallskipamount

\sloppy


%%%%%%%%%%%%%%%%%%%%%%%%%%%%%%%%%%%%%%%%%%%%%%%%%%%%%%%%%%%%%%%%%%%%%

%% handy commands

\newcommand{\eqdf}{\stackrel{\scriptscriptstyle \triangle}{=}}
\newcommand{\argmin}{\ensuremath{\mbox{argmin}}}
\newcommand{\argmax}{\ensuremath{\mbox{argmax}}}
\newcommand{\set}[1]{\left\{ #1 \right\}}
\newcommand{\paren}[1]{\left( #1 \right)}
\newcommand{\bit}{\set{0,1}}
\newcommand{\inv}[1]{\frac{1}{#1}}
\newcommand{\abs}[1]{\left| #1 \right|}
\newcommand{\ceil}[1]{\left\lceil {#1} \right\rceil}
\newcommand{\floor}[1]{\left\lfloor {#1} \right\rfloor}
\newcommand{\half}{\frac{1}{2}}
\newcommand{\threehalves}{\frac{3}{2}}

\newcommand{\naturals}{\mathbb{N}}
\newcommand{\rationals}{\mathbb{Q}}
\newcommand{\reals}{\mathbb{R}}
\newcommand{\integers}{\mathbb{Z}}

\newcommand{\eps}{\varepsilon}

%%%%%%%%%%%%%%%%%%%%%%%%%%%%%%%%%%%%%%%%%%%%%%%%%%%%%%%%%%%%%%%%%%%%%

\newcommand{\scp}{\textsc{SCP}\xspace}
\newcommand{\scplong}{\textsc{Service Chain Placement}\xspace}

\newcommand{\calE}{\mathcal{E}}
\newcommand{\calG}{\mathcal{G}}
\newcommand{\calV}{\mathcal{V}}

\newcommand{\opt}{\textsc{opt}\xspace}
\newcommand{\cost}{\textsc{cost}\xspace}
\newcommand{\process}{\textsc{prc}\xspace}
\newcommand{\latency}{\textsc{lat}\xspace}

\newcommand{\dror}[1]{\sethlcolor{yellow}\hl{Dror: #1}}

%%%%%%%%%%%%%%%%%%%%%%%%%%%%%%%%%%%%%%%%%%%%%%%%%%%%%%%%%%%%%%%%%%%%%

\begin{document}

\title{\bf Service Chain Placement in SDNs%
\thanks{Research supported in part by Network Programming (Neptune)
  Consortium, Israel.}
}

\titlerunning{Service Chain Placement in SDNs}


%\author[1]{Gilad Kutiel%
%\thanks{E-mail: \texttt{gkutiel@cs.technion.ac.il}}}

%\author[2]{Dror Rawitz%
%\thanks{Partially supported by the Israel Science Foundation
%  (grant no.~497/14).  E-mail: \texttt{dror.rawitz@biu.ac.il}}}

%\affil[1]{Department of Computer Science, Technion, Haifa, Israel}
%\affil[2]{Faculty of Engineering, Bar Ilan University, Ramt-Gan, Israel}

\author{%
Gilad Kutiel
\and
Dror Rawitz%
\thanks{Partially supported by the Israel Science Foundation
  (grant no.~497/14).  }
}

\institute{%
Department of Computer Science, Technion, Haifa 32000, Israel. \\
\email{gkutiel@cs.technion.ac.il}
\and
Faculty of Engineering, Bar Ilan University, Ramt-Gan 52900, Israel.\\
\email{dror.rawitz@biu.ac.il}
}

%%%%%%%%%%%%%%%%%%%%%%%%%%%%%%%%%%%%%%%%%%%%%%%%%%%%%%%%%%%%%%%%%%%%%

\maketitle

\begin{abstract}
We study the allocation problem of a compound request for a
\emph{service chain} in a software defined network that supports
network function virtualization.
%
Given a network that contains servers with limited processing power
and links with limited bandwidth, a service chain is a sequence of
virtual network functions (VNFs) that serve a certain flow request
in the network.
%
The allocation of a service chain consists of routing and VNF
placement, namely each VNF from the sequence is placed in a server
along a path.  It is feasible if each server can handle the VNFs that
are assigned to it, and if each link on the path can carry the flow
that is assigned to it.
%
A request for service is composed of a source and a destination in the
network, an upper bound on the total latency, and a specification in
the form of a directed acyclic graph (DAG) of VNFs that provides all
service chains that are considered valid for this request.
%
In addition, each pair of server and VNF is associated with a cost for
placing the VNF in the server.  Given a request, the goal is to find a
valid service chain of minimum total cost that respects the latency
constraint or to identify that such a service chain does not exist.

We show that even the feasibility problem is NP-hard in general
graphs.  Hence we focus on DAGs.  We show that the problem is still
NP-hard in DAGs even for a very simple network, and even if the VNF
specification consists of only one option (i.e., the virtual DAG is a
path).
%
On the other hand, we present an FPTAS for the case where the network
is a DAG.  In addition, based on our FPTAS, we provide randomized
algorithms for instances in which the service chain passes through a
bounded number of vertices whose degree is larger than two.
\end{abstract}


% Keywords:
% approximation algorithms,
% randomized algorithms,
% network function virtualization,
% resource allocation,
% service chaining,
% software defined network,
% 

%%%%%%%%%%%%%%%%%%%%%%%%%%%%%%%%%%%%%%%%%%%%%%%%%%%%%%%%%%%%%%%%%%%%%

\section{Introduction}


Computer communication networks are in constant need of expansion to
cope with the ever growing traffic.  As networks grow, management and
maintenance become more and more complicated.
%
Current developments that aim to improve the utilization of network
resources include the detachment of network applications from network
infrastructure and the transition from network planning to network
programming.

One aspect of network programming is to manage resources from a
central point of view, namely to make decisions based on availability,
network status, required quality of service, and the identity of the
client.  Hence, a focal issue is a central agent that is able to
received reports from network components and client requests, and as a
result can alter the allocation of resources in the networks.  This
approach is called Software Defined Networking (SDN), where there is a
separation between routing and management (control plane) and the
underlying routers and switches that forward traffic (data plane) (see
Kreutz et al.~\cite{KRVRAU15}).

A complementary approach is Network Function Virtualization
(NFV)~\cite{NFV12}.  Instead of relying on special purpose machines,
network applications become virtual network functions (VNF) that are
executed on generic machines and can be placed in various locations in
the network.  Virtualization increases the flexibility of resource
allocation and thus the utilization of the network resources.
%
Internet Service Providers (ISPs) that provide services to clients
benefit from NFV, since it helps to better utilize their physical
network.  In addition, when network services are virtualized, an ISP
may support \emph{service chaining}~\cite{ServiceChaining15}, namely a
compound service that consists of a sequence of VNFs.  Furthermore,
one may offer a compound service a service that may be obtained using
one of several sequences of VNFs.

%%%%%

\subsection{Related Work}
%\paragraph*{\bf Related work.}

In this paper we consider an SDN network that employs
NVF~\cite{SDN-NFV15}.  Such networks attracted a lot of attention in
the networking community, especially from a systems point of view
(see, e.g.,~\cite{GVPGKDA14,HVSBFTF15}).  Several papers also consider
the algorithmic aspects of such networks but mainly present heuristics
(see, e.g,~\cite{soule2014merlin}).  
For more details see~\cite{ERS16} and references therein.

Cohen et al.~\cite{CLNR15} considered VNF placement.  In their model
the input is an undirected graph with a metric distance function on
the edges.  Clients are located in various nodes, and each client is
interested in a subset of VNFs.  For each node $v$ and VNF $\alpha$,
there is a setup cost associated with placing a copy of $\alpha$ at
$v$ (multiple copies are allowed), and there is a load that is induced
on $v$ for placing a copy of $\alpha$.  Furthermore, each node has a
limited capacity, and each copy of a VNF can handle a limited amount
of clients.  A solution is the assignment of each client to a subset
of nodes, each corresponding to one of its required VNFs.  The cost of
a solution is the total setup costs plus the sum of distances between
the clients and the node from which they get service.
%
Cohen et al.~\cite{CLNR15} gave bi-criteria approximation algorithms
for various versions of the problem, namely algorithms that compute
constant factor approximations that violate the load constraints by a
constant factor.
%
It is important to note that in this problem routing is not considered
and it is assumed that VNF subsets can be executed in any order.

Lukovszki and Schmid~\cite{LukovszkiSchmid15} studied the problem of
admission control and placement of service chains, where each chain is
associated with a source-destination pair and is supposed to be routed
via an ordered sequence of $\ell$ VNFs.  The VNFs may have multiple
instantiations, but each node has a limited capacity that bounds the
number of requests it may service.
%
They presented an $O(\log \ell)$-competitive algorithm for the problem
of maximizing the number of serviced chains, assuming that capacities
are $\Omega(\log \ell)$.
%at least logarithmic in $\ell$.
It is also shown that this ratio is asymptotically optimal even for
randomized online algorithms.  APX-hardness results for the offline
version of the problem were also presented.
%
Rost and Schmid~\cite{RostSchmid16} considered variant of this
problem, where each node can host a subset of the VNFs.  In addition,
each VNF has a demand and each VNF-node pair has a capacity, and in a
feasible solution the total demand for each pair is bounded by the
capacity.  They considered two goals: maximum profit and minimum
resource cost and gave bicriteria approximation algorithms which are
based on LP-rounding for several special cases.

Even, Medina, and Patt-Shamir~\cite{EMP16} studied online path
computation and VNF placement.  They considered compound requests that
arrive in an online manner.  Each request is a flow with a
specification of routing and VNF requirements which is represented by
a directed acyclic graph (DAG) whose vertices are VNFs.  Each VNF can
be performed by a specified subset of servers in the system.  Upon
arrival of a request, the algorithm either rejects the request or
accepts it with a specific routing and VNF assignment, under given
server capacity constraints.  Each request has a benefit that is
gained if it is accepted, and the goal is to maximize the total
benefit gained by accepted requests.
%
Even, Medina, and Patt-Shamir~\cite{EMP16} proposed an algorithm that
copes with requests with unknown duration without preemption by using
a third response, which is refer to as “stand by”, whose competitive
ratio is $O(\log (knb_{\max}))$, where $n$ is the number of nodes,
$b_{\max}$ is an upper bound on a request benefit per time unit, and
$k$ is the maximum number of VNFs of any service chain.  This
performance guarantee holds for the case where the processing of any
request in any possible service chain is never more than
$O(1/(k \log (nk)))$ fraction of the capacity of any network
component.
%
Even, Rost, and Schmid~\cite{ERS16} presented a randomized
approximation algorithm for the same problem that computes
$(1-\eps)$-approximate placements if the demand of any processing
request in any possible service chain is never more than an
$O(\eps^2/\log n)$ fraction of the capacity of any network component.

%%%%%

\subsection{Our Results}
%\paragraph*{\bf Our results.}

We study the allocation problem of a compound flow request for a
service chain in a software-defined network that supports NFV, where
the goal is to find an optimal placement with respect to the current
available resources from the point of view of the network (i.e., the
ISP).
%
More specifically, we consider the case where the network already
contains previous resource allocations.  Therefore, we are given a
(physical) network that contains servers with limited (residual)
processing power and links with limited (residual) bandwidth.
%
A request for service is composed of a source and a destination in the
network, an upper bound on the total latency, and a specification of
all service chains that are considered valid for this request.  As
in~\cite{EMP16}, the specification is represented by a DAG whose
vertices are VNFs.
%
The allocation of a service chain consists of routing and VNF
placement.  That is, each VNF from the sequence is placed in a server
along a path, and it is feasible if each server can handle the VNFs
that are assigned to it, and if each link on the path can carry the
flow that is assigned to it.  Moreover each link causes a delay, and
the service chain placement should also comply with a global bound on
the total latency.
%
Each pair of server and VNF is associated with a cost for placing the
network function in the server.  This cost measures compatibility of a
VNF to a server (e.g., infinite costs means ``incompatible'').  Given
a request, the goal is to find a feasible service chain of minimum
total cost or to identify that a valid service chain does not exist.

We show that even feasibility is NP-hard in general networks using a
reduction from \textsc{Hamiltonian Path}.
%Hence we focus on DAGs.
We show that the problem is still NP-hard if the network is a DAG even
for a very simple network, and even if the specification consists of a
single option.  Both reductions are from \textsc{Partition}.
%
On the positive side, we present an FPTAS for the case where the
network is a DAG which is based on a non-trivial dynamic programming
algorithm.
%
Based on our FPTAS, we provide a randomized algorithm for general
networks in which there is an optimal placement whose physical path
consists of at most $k$ vertices whose degree is larger than $2$.  For
example, this can be assumed if all simple paths from $s$ to $t$
contain at most $k$ such vertices.  The algorithm computes a
$(1+\eps)$-approximate placement with high probability using $k! \log
n$ invocations of the FPTAS.
%
We also present a (deterministic) parameterized algorithm that
computes a $(1+\eps)$-approximate placement in time $O(k! \cdot
\text{poly}(n))$, where $k$ is the number of vertices in the network
whose degree is larger than $2$.


%%%%%%%%%%%%%%%%%%%%%%%%%%%%%%%%%%%%%%%%%%%%%%%%%%%%%%%%%%%%%%%%%%%%%

\section{Preliminaries}

\subsection*{Model.}
%
An instance of the \scplong (\scp) problem is composed of three
components, a physical network, a virtual specification, and placement
costs:
\begin{description}
\item[Physical network:]
%
  The physical network is a graph $G = (V,E)$.  Each node $v \in V$
  has a non-negative processing capacity $p(v)$, and each directed
  edge $e \in E$ has a non-negative bandwidth capacity $b(e)$.
%
  Without loss of generality, we assume that $p(s) = p(t) = 0$.  

\medskip
  
\item[Virtual specification:]
%
  The description of a request for a service chain consists of a
  physical source $s \in V$, a physical destination $t \in V$, and a
  directed acyclic graph (DAG) $\calG = (\calV,\calE)$.
%
  Without loss of generality, we assume that $p(s) = p(t) = 0$.  
%
  The DAG $\calG$ has a source node $\sigma \in \calV$ and a
  destination $\tau \in \calV$. Each node $\alpha \in \calV$ represent
  a VNF that has a processing demand $p(\alpha)$.
%
  Without loss of generality, we assume that $p(\sigma) = p(\tau) =
  0$.  Each edge $\eps \in \calE$ has a bandwidth demand $b(\eps)$.

\medskip
  
\item[Placement costs:]
%
  There is a non-negative cost $c(\alpha,v)$ for placing the VNF
  $\alpha$ in $v$.  We assume that $c(\sigma,s) = 0$ and $c(\sigma,v)
  = \infty$, for every $v \neq s$.  Similarly, we assume that
  $c(\tau,t) = 0$ and $c(\tau,v) = \infty$, for every $v \neq t$.
\end{description}

A solution consists of the following:
\begin{description}
\item[Virtual path:] A path from $\sigma$ to $\tau$ in $\calG$, namely
  a sequence of vertices $\alpha_0,\ldots,\alpha_q$, where
  $\alpha_0 = \sigma$, $\alpha_q = \tau$, and
  $(\alpha_j,\alpha_{j+1}) \in \calE$, for every $j \in
  \set{0,\ldots,q-1}$.

\medskip
  
\item[Physical path:] A simple path from $s$ to $t$ in $G$, namely a
  sequence of nodes $v_0,\ldots,v_k$, where $v_0 = s$, $v_k = t$, and
  $(v_i,v_{i+1}) \in E$, for every $i \in \set{0,\ldots,k-1}$.

\medskip
  
\item[Placement:] A function $f$ that maps a virtual path to a simple
  physical path.  Formally, a placement is a function $f:
  \set{\alpha_0,\ldots,\alpha_q} \to \set{v_0,\ldots,v_k}$ where 
\begin{enumerate}
\item $v_i \neq v_{i'}$ if $i \neq i'$.    
\item If $f(\alpha_j) = v_i$ and $f(\alpha_{j+1}) = v_{i'}$, then $i
  \leq i'$.
\item $b(\alpha_j,\alpha_{j+1}) \leq b(v_i,v_{i+1})$, for every
  $(v_i,v_{i+1}) \in \bar{f}(\alpha_j,\alpha_{j+1})$, where
  $\bar{f}(\alpha_j,\alpha_{j+1})$ to be the set of physical edges
  that correspond to it, i.e.,
\[
\bar{f}(\alpha_j,\alpha_{j+1}) \eqdf \set{(v_i,v_{i+1}) : i' \leq i < i''}
~,
\]
  where $f(\alpha_j) = v_{i'}$ and $f(\alpha_{j+1}) = v_{i''}$.

\item $\sum_{\alpha \in f^{-1}(v)} p(\alpha) \leq p(v)$, where
  $f^{-1}(v) \eqdf \set{\alpha : f(\alpha) = v}$.
 
\end{enumerate}
\end{description}

An example of an \scp instance and a solution is given in
Figure~\ref{fig:solution}.

\begin{figure}[t]
\centering
\scalebox{0.85}{
\begin{tikzpicture}[every node/.style={default node}, very thick]

\begin{scope}[random seed=1, spring layout, node distance=1.8cm]
\node(a) {a};
\node(b) {b};
\node(c) {c};
\node(d) {d};
\node(e) {e};
\node(f) {f};
\node(g) {g};
\node(h) {h};
\node(i) {i};
\node(j) {j};

\graph{
	(a) -> (b);
	(b) -> {(d), (e)};
	(d) -> {(f), (h)};
	(f) -> {(g), (h), (c)};
	(h) -> {(g), (i), (e)};
	{[edges={blue, dashed}]
		(a) -> (c) -> (d) -> (e) -> (i) -> (g) -> (j);
	}
};

\node(pa) {a'};
\node(pb) {b'};
\node(pc) {c'};
\node(pd) {d'};
\node(pe) {e'};
\node(pf) {f'};
\node(pg) {g'};
\node(ph) {h'};

\graph{
	(pa) -> (pb);
	(pb) -> {(pd), (pe)};
	(pd) -> {(pf), (ph)};
	(pf) -> {(pg), (ph), (pc)};
	(ph) -> {(pg), (pe)};
	(pe) -> (ph);
};
\end{scope}

\begin{scope}[dotted, orange, -]
\draw
(a.north east) to[out=135, in=45] node(j1)[midway, clear node] {}
(c.north west) to[out=225, in=135]
(d.south west) to[out=-45, in=-45]
(a.north east)
;

\draw
(e.east) to[out=-90, in=-90] node(j2)[midway, clear node] {}
(i.west) to[out=90, in=90]
(e.east)
;

\draw
(g.east) to[out=-90, in=-90] node(j3)[midway, clear node] {}
(j.west) to[out=90, in=90]
(g.east)
;

\draw[->] (j1) to[bend left] (pa);
\draw[->] (j2) to[bend right] (pe);
\draw[->] (j3) to[bend right] (pg);
\end{scope}
	

\node[clear node] at(-5,0) {Virtual Graph};
\node[clear node] at(5, -2){Physical Graph};

\end{tikzpicture}
}
\caption{An example of an \scp instance and a solution.}
\label{fig:solution}
\end{figure}

The cost of a placement $f$ is defined as:
\[
c(f) \eqdf \sum_j c(\alpha_j,f(\alpha_j))
~.
\]
In \scp the goal is to find a feasible placement of a virtual path
into the physical DAG that minimizes the cost.

%%%%%

We also consider an extended version of \scp in which each physical
link causes a delay and there is an upper bound $L$ on the total
delay.  More formally, each pair of a virtual edge $\eps$ and physical
edge $e$ is associated with a latency $\ell(\eps,e)$.
%
Given a placement $f: \set{\alpha_0,\ldots,\alpha_q} \to
\set{v_0,\ldots,v_k}$, the total latency of the solution is given by
\[
L(f)
= \sum_{j=0}^{q-1} \sum_{e \in \bar{f}((\alpha_j,\alpha_{j+1}))} \ell((\alpha_j,\alpha_{j+1}),e)
~.
\]
In this case there is an additional constraint that $L(f) \leq L$.

%We note that this global bound does not have to represent latency.
%For example, it can bound the probability of failure along the path.
%(In this $\ell(\eps,e)$ is the logarithm of the probability.)

%%%%%

%\paragraph*{\bf Definitions and notation.}
%
%Define $n_p \eqdf \abs{V}$ and $m_p \eqdf \abs{E}$.  Similarly, define
%$n_v \eqdf \abs{\calV}$ and $m_v \eqdf \abs{\calE}$.

\subsection*{Notation.}
%
Given a virtual DAG $\calG$, we assume the existence of a topological
sorting of the vertices and write $\alpha \prec \beta$ if $\alpha$
precedes $\beta$ in this ordering.  If the physical network is a DAG,
then we make a similar assumption and write $v \prec u$ if $v$
precedes $u$ in this ordering.


%%%%%%%%%%%%%%%%%%%%%%%%%%%%%%%%%%%%%%%%%%%%%%%%%%%%%%%%%%%%%%%%%%%%%

\section{Hardness Results}

In this section we present three hardness results.  First, we show
that even the feasibility question of \scp is NP-hard, and therefore
no approximation algorithm exists for \scp.  In addition, we show that
\scp is NP-hard even if $\abs{V \setminus \set{s,t}} = 1$.  We also
show that \scp is NP-hard even if both the physical network and the
virtual DAG are simple paths.

We start by showing that even finding a feasible solution is NP-hard.

\begin{theorem}
Feasibility of \scp is NP-hard
\end{theorem}
\begin{proof}
We use a reduction from \textsc{Hamiltonian Path} that is known to be
NP-hard~\cite{GarJoh79}.
%
Given an instance $H$ of \textsc{Hamiltonian Path} we construct an
instance of \scp.  For the physical network we have that $G$, where
$V(G) = V(H) \cup \set{s,t}$, and $E(G) = E(H) \cup \set{(s,v),(v,t) :
  v \in V(H)}$.  In addition, $p(v) = 1$, for every $v$, and $b(e) =
1$, for every $e$.  The virtual DAG $\calG$ is a path containing $n+2$
virtual functions, $\sigma = \alpha_0, \alpha_1, \ldots, \alpha_n,
\alpha_{n+1} = \tau$, where $p(\alpha_i) = 1$, for every $i \in
\set{1,\ldots,n}$.  Also, $b(\alpha_i,\alpha_{i+1}) = 1$, for every
$i$.

The construction can clearly be computed in polynomial time.
%
An Hamiltonian Path in $G$, induces a \scp solution that follows the
path, i.e., $\alpha_i$ is placed in the $i$th vertex along the path.
%
On the other hand, since all demands and capacities are $1$, no two
VNFs can share a physical node.  Hence a \scp solution induces an
Hamiltonian path.
%
\qed
\end{proof}

Next, we show that \scp is NP-hard even if the physical network is
very simple. 

\begin{theorem}
\label{thm:simple}
\scp is NP-hard, even if $\abs{V \setminus \set{s,t}} = 1$.
\end{theorem}
\begin{proof}
We prove the theorem using a reduction from \textsc{Partition}.
Given a \textsc{Partition} instance $\set{a_1, \ldots, a_n}$, we
construct an \scp instance as follows.
%
The physical network $G$ contains three nodes: $s$, $v$, and $t$, and
there are two edges $(s,v)$ and $(v,t)$.  The capacity of $v$ is $p(v)
= \half \sum_i a_i$.  Edge bandwidth are zero.
%
The virtual DAG is composed of $2n+2$ vertices, namely
\[
\calV
= \set{\sigma,\tau}
  \cup \set{\alpha_1, \ldots, \alpha_n}
  \cup \set{\beta_1, \ldots, \beta_n}
~.
\]
Also,
\[
\calE
= \bigcup_i \set{ \set{\alpha_i,\beta_i} \times \set{\alpha_{i+1},\beta_{i+1}} } 
  \cup
  \set{ (\sigma,\alpha_1), (\sigma,\beta_1), (\alpha_n,t), (\beta_n,t) }
~.
\]
The DAG is shown in Figure~\ref{fig:simple}.
%
The demands are $p(\alpha_i) = a_i$ and $p(\beta_i) = 0$, for every
$i$.  Also, $b(\eps) = 0$, for every $\eps \in \calE$.  The costs are:
$c(\alpha_i,v) = 0$ and $c(\beta_i,v) = a_i$, for every $i$.

The \scp instance can be computed in polynomial time.
%
Also, it is not hard to verify that $\set{a_1, \ldots, a_n} \in
\textsc{Partition}$ if and only if there exists a solution whose cost
is $\half \sum_i a_i$.
%
\qed
\end{proof}

\begin{figure}[t]
  \centering
\scalebox{1}{
  \begin{tikzpicture}[every node/.style={default node}]

\begin{scope}[layered layout, grow=right, sibling distance=2cm, level distance=2cm]

\node(v1p) {$\beta_1$};
\node(v1m) {$\alpha_1$};
\node(v2p) {$\beta_2$};
\node(v2m) {$\alpha_2$};
\node[clear node, inner sep=5mm](dotsp) {$\ldots$};
\node[clear node, inner sep=5mm](dotsm) {$\ldots$};
\node(vnp) {$\beta_n$};
\node(vnm) {$\alpha_n$};

\graph{
	s -> {(v1p), (v1m)};
	(v1p) -> {(v2p), (v2m)};
	(v1m) -> {(v2p), (v2m)};
	(v2p) -> {(dotsp), (dotsm)};
	(v2m) -> {(dotsp), (dotsm)};
	(dotsp) -> {(vnp), (vnm)};
	(dotsm) -> {(vnp), (vnm)};
	{(vnp), (vnm)} -> t;
};

\end{scope}

\end{tikzpicture}

}
\caption{The specification defined in the proof of Theorem~\ref{thm:simple}.}
\label{fig:simple}
\end{figure}

Next, we show that \scp is NP-hard even if both the physical network
and the VNF specification are paths.

\begin{theorem}
\scp is NP-hard, even if both the physical network and the virtual DAG
are paths.
\end{theorem}
\begin{proof}
We prove the theorem using a reduction from \textsc{Partition}.  Given
a \textsc{Partition} instance $\set{a_1, \ldots, a_n}$, we construct a
virtual path $\sigma, \alpha_1, \ldots, \alpha_n, \tau$ and a physical
path $s, v_1^-, v_1^+, \ldots, v_n^-, v_n^+, t$.  We set $p(v) = 1$,
for every node $v \neq s,t$, and $p(\alpha_i) = 1$, for every $i$.  In
addition, we set $b(e) = 1$, for every edge $e \in E$, and $b(\eps) =
1$, for every edge $\eps \in \calE$.
%
As for the costs, we define $c(\alpha_i, v_i^-) = 0$, $c(\alpha_i,
v_i^+) = a_i$, and for any $v \notin \{v_i^-,v_i^+\}$ we set
$c(\alpha_i, v) = \infty$.  We also define $\ell((\alpha_i,\alpha_{i+1}),
(v_i^-,v_i^+)) = a_i$, and set $\ell(\eps,e) = 0$, otherwise.  Finally,
we set $L = \half \sum_i^n a_i$.
%
Figure~\ref{fig:reduction2} depicts the above reduction.
%
One can verify that $a_i$ is either counted in the latency of in the
cost.  Hence, $\set{a_1, \ldots, a_n} \in \textsc{Partition}$ if and
only if there is a placement with cost $\half \sum_i^n a_i$.
%
\qed
\end{proof}


\begin{figure}[t]
\centering
\scalebox{.75}{
\begin{tikzpicture}[every node/.style={default node}]
%\node[draw=none] at(3, 1) {Virtual Path};

\node(sigma) at (-2,0) {$\sigma$};
\node(alpha1) at(0,0) {$\alpha_1$};
\node(alpha2) at(2,0) {$\alpha_2$};
\node[draw=none] at(6,0) {$\cdots$};
\node(alphan) at(10,0) {$\alpha_n$};
\node(tau) at(12,0) {$\tau$};

\draw (sigma) -- (alpha1);
\draw (alpha1) -- (alpha2);
\draw (alpha2) -- +(2, 0);
\draw ($(alphan) + (-2,0)$) -- (alphan);
\draw (alphan) -- (tau);

\node(s) at (-2,-4) {$s$};
\node(v1-) at(0, -2) {$v_1^-$};
\node(v1+) at(0, -4) {$v_1^+$};
\node(v2-) at(2, -2) {$v_2^-$};
\node(v2+) at(2, -4) {$v_2^+$};

\node[draw=none] at(6,-2) {$\cdots$};
\node[draw=none] at(6,-4) {$\cdots$};

\node(vn-) at(10, -2) {$v_n^-$};
\node(vn+) at(10, -4) {$v_n^+$};
\node(t) at(12, -2) {$t$};

\draw (s) -- (v1-) node[label] {$0$};
\draw (v1-) -- (v1+) node[label] {$a_1$};
\draw (v1+) -- (v2-) node[label] {$0$};
\draw (v2-) -- (v2+) node[label] {$a_2$};
\draw (v2+) -- +(2, 2)node[label] {$0$};

\draw ($(vn-) + (-2,-2)$) -- (vn-)node[label] {$0$};;
\draw (vn-) -- (vn+)node[label] {$a_n$};
\draw (vn+) -- (t)node[label] {$0$};;

\begin{scope}[every node/.style={label}, dotted, purple, ->]
\draw (sigma) to[bend right] node {$0$} (s);

\draw (alpha1) to[bend left] node {$0$} (v1-);
\draw (alpha1) to[bend right] node {$a_1$} (v1+);
\draw (alpha2) to[bend left] node {$0$} (v2-);
\draw (alpha2) to[bend right] node {$a_2$} (v2+);
\draw (alphan) to[bend left] node {$0$} (vn-);
\draw (alphan) to[bend right] node {$a_n$} (vn+);

\draw (tau) to[bend left] node {$0$} (t);
\end{scope}

\end{tikzpicture}

}
\caption{A reduction from a partition instance $a_1, \ldots, a_n$.
  The cost function enforces each $\alpha_i$ to be placed either in
  $v_i^-$ or in $v_i^+$.  In the former case this placement incure no
  cost but additional latency of $a_i$, in the later case there will
  be additional $a_i$ cost with zero latency.}
\label{fig:reduction2}
\end{figure}


%%%%%%%%%%%%%%%%%%%%%%%%%%%%%%%%%%%%%%%%%%%%%%%%%%%%%%%%%%%%%%%%%%%%%

\section{Algorithms for Physical Directed Acyclic Graphs}
\label{sec:algorithms}

In this section we present an FPTAS for \scp in DAGs which is based on
a dynamic programming algorithm.
%
The algorithm is described in a top down manner.  We first assume that
costs are polynomially bounded and design a dynamic programming
algorithm that computes a minimum cost placement of a virtual path
within a single physical node.  Next, we provide a dynamic programming
algorithm for \scp without the latency constraint.  Then, we give an
algorithm that copes with a global latency constraint.  At the end of
the section we drop our assumption on the costs and use standard
scaling techniques to obtain an FPTAS for \scp in DAGs.

%%%%%

\subsection{Placing a Sub-chain in a Physical Node}

Assume that we want to place a minimum cost virtual path from a VNF
$\alpha$ to a VNF $\beta$ within a physical node $v$.  A sequence of
VNFs $\alpha = \alpha_0,\alpha_2,\ldots,\alpha_q = \beta$ is a
\emph{candidate path} if $(\alpha_i,\alpha_{i+1}) \in \calE$, for
every $i$, and $\sum_i p(\alpha_i) \leq p(v)$.  We would like to find
such a path with minimum cost, and we denote the cost of such a path
by $\cost_v(\alpha \leadsto \beta)$.

We use dynamic programming in order to compute such a path.  Let
$\process_v(\alpha \leadsto \beta,c)$ be the minimum amount of
processing required to place a virtual path form $\alpha$ to $\beta$
into $v$ among paths whose cost is at most $c$.
%
The value of $\process_v(\alpha \leadsto \beta,c)$ can be computed
recursively as follow:
\[
\process_v(\alpha \leadsto \beta, c) =
\begin{cases}
%\infty    & c < 0, \\
\infty    & c(\alpha,v) > c, \\
p(\alpha) &
\begin{array}{c}
c(\alpha,v) \leq c, \\
\beta = \alpha,
\end{array}
\\
\displaystyle
\min_{(\gamma,\beta) \in \calE}
   \set{\process_v(\alpha \leadsto \gamma, c - c(\beta,v)) + p(\beta)}
          & \text{otherwise}.
\end{cases}
\]
Observe that if $c(\alpha,v) > c$, then a placement is not possible
with a budget $c$.  Otherwise, if $\alpha = \beta$, then the best path
is the one containing only $\alpha$.  If $\alpha \neq \beta$, then an
optimal path ends with an edge $(\gamma,\beta) \in \calE$, which means
that the processing placed at $v$ consists of $p(\beta)$ plus the
minimum amount of processing of a path from $\alpha$ to $\gamma$,
whose cost is at most $c - c(\beta,v)$.

To complete our argument observe that the following holds:
\[
\cost_v(\alpha \leadsto \beta)
= \min \set{c : \process_v(\alpha \leadsto \beta, c) \leq p(v)}
~.
\]

Since the costs are assumed to be polynomially bounded, there is a
polynomial number of states to be computed, and the computation of
each of them can be done in linear time.  Hence, the total running
time is polynomial.  Finally, we note that the above algorithm
computes the minimum amount of processing, but may also be used to
compute the actual placement that achieves this value using standard
techniques.

%%%%%

\subsection{Placing a Service Chain}

In this section we describe a dynamic programming algorithm for
placing a service chain without a global latency bound.  

Consider an optimal solution of \scp, i.e., a service chain from
$\sigma$ to $\tau$ which is placed along a path from $s$ to $t$ in the
physical DAG.  Suppose $v$ is a node along the path from $s$ to $t$.
Also, let $\alpha$ be the last VNF in the service chain which is
placed in the path from $s$ to $v$.  It must be that the placement of
the VNFs from $\sigma$ to $\alpha$ in the path from $s$ to $v$ is the
best one among all placements of a virtual path from $\sigma$ to
$\alpha$ in a path from $s$ to $v$.  In other words, any partial
placement of an optimal placement is also optimal.  Our algorithm is
based on this property.

We define a state for each pair of VNF $\alpha$ and physical node $v$.
Let $\cost(\alpha,v)$ stand for the minimum cost placement of a
virtual path from $\sigma$ to $\alpha$ in a path from $s$ to $v$,
where $\alpha$ is the last VNF which is placed along the physical
path.
%
In addition, we write $(u, v) \Mapsto (\gamma,\beta)$ if $v$ is
reachable from $u$ using only edges with bandwidth at least
$b(\gamma,\beta)$, namely if there is a path from $u$ to $v$ such that
the bandwidth of all edges in the path is at least $b(\gamma,\beta)$.
%
$\cost(\alpha,v)$ can be computed recursively, as follows:
\[
\cost(\alpha,v) =
\begin{cases}
0 & \alpha = \sigma, v = s, \\
\displaystyle
\min_{
  \substack{\beta \prec \alpha, (\gamma,\beta) \in \calE, \\
           u \prec v, \\
         (u,v) \Mapsto (\gamma,\beta)}
     }
  \set{\cost(\gamma,u) + \cost_v(\beta \leadsto \alpha)}
  & \text{otherwise}.
\end{cases}
\]
The desired value is $\cost(\tau,t)$.
%
Consider an optimal placement of a virtual path ending at $\alpha$
within a path ending at $v$.  Let $\gamma$ be the first VNF along the
virtual path that is not placed in $v$.  Also, let $u$ be the node in
which $\gamma$ is placed.  Hence, the optimal placement is composed of
three segments: an optimal placement of a virtual path that ends in
$\gamma$ in a physical path ending at $u$, a virtual edge
$(\gamma,\beta)$ is placed in the path from $u$ to $v$, and a minimum
cost placement of a virtual path from $\beta$ to $\alpha$ in $v$, that
does not violate the capacity of $v$.
%
Thus, we check all pairs $(\gamma,u)$, where $\gamma \prec \alpha$ and
$u \prec v$, and for each pair we consider any neighbor
$(\gamma,\beta) \in \calE$ and $\beta \prec \alpha$ such that $(u, v)
\Mapsto (\gamma,\beta)$.
%
The recursive computation is illustrated in Figure~\ref{fig:dp1}.

\begin{figure}[t]
\centering
\scalebox{0.9}{
\begin{tikzpicture}[every node/.style={default node}]
\node[draw=none] at(5, 2.7) {Virtual Graph};

\node(sv) at(0,0) {$\sigma$};
\node(v0) at(4,0) {$\alpha$};
\node(v1) at(6.5,1) {$\beta$};
\node[draw=none] at(6.5,0) {$\vdots$};
\node(v3) at(6.5,-1) {};
\node(tv) at(10,0) {$\gamma$};

\draw[cloud, blue, dashed] 
($(sv.west) + (-.3,0)$)	to[out=90, in=90] 
($(v0.east) + (.3, 0)$)	to[out=-90, in=-90]
($(sv.west) + (-.3,0)$)
;

\draw (v0) -- (v1) node[label above] {$b_d(\alpha\beta)$};
\draw[dashed] (v0) -- (v3);

\draw[path] (v1) -- (tv.north west);
\draw[path] (v3) -- (tv.south west);

\draw[brace, light] ($(v0) + (-.5, 1.6)$) -- ($(tv) + (.5, 1.6)$)
node[label above, above=2mm] {$P_{v}(\alpha , \gamma, C)$}
;


\node[draw=none] at(4.5, -8) {Physical Graph};
\node(s) at(0,-5) {s};
\node(p1) at(5, -4){u};
\node[draw=none] at(5, -5){$\vdots$};
\node(p3) at(5, -6){};
\node(t) at(7.5,-5) {v};

\draw[cloud, blue, dashed] 
($(s.west) + (-.3,0)$)			to[out=85, in=150] 
($(p1.north east) + (.3, 0)$)	to[out=-90, in=90]
($(p3.south east) + (.3, 0)$)	to[out=210, in=-85]
($(s.west) + (-.3,0)$)
;

\draw (p1) -- (t) node[label above] {$b_c(uv)$};
\draw (p3) -- (t);

\draw[group]
(v1.west) to[out=90, in=180]
(v1.north) to[out=0, in=135]
(tv.north east) to[out=-45, in=0]
(tv.south) to[out=180, in=-45] coordinate[near start](f)
(v1.south west) to[out=135, in=-90]
(v1.west)
;

\draw[group, ->] (f) to[bend left] (t.east);
\end{tikzpicture}
}
%\vspace{-20pt}
\caption{The optimal placement of a path to $\alpha$ into a path to
  $v$ can be efficiently computed by breaking the problem into the
  problem of placing a path to $\gamma$-path into a path to $u$ (blue,
  dashed rectangles) and the problem of embedding a path from $\beta$
  to $\alpha$ into $v$ (orange, dotted).  The path from $u$ to $v$
  must carry at least $b(\gamma,\beta)$ bandwidth.  }
\label{fig:dp1}
\end{figure}

As for the running time, observe that checking whether $(u, v) \Mapsto
(\gamma,\beta)$ can be done using DFS in linear time.
%
Since the number of state is polynomial, and each state can be
computed in polynomial time, the total running time is polynomial.
Finally, we note that the above algorithm computes the minimum amount
of cost, but may also be used to compute the actual placement
that achieves this value using standard techniques.

%%%%%

\subsection{Placing a Service Chain with a Latency Bound}
\label{sub:DPL}

We now consider the \scp problem with latency.  Recall that in this
variant of the problem we are also given a latency function $\ell: E
\times \calE \to \reals_+$, and a latency upper bound $L$.  The goal
is to find a minimum cost placement that also respect the latency
constraint.

Let $\latency(\alpha,v,c)$ be the minimum latency created by a
placement of a virtual path from $\sigma$ to $\alpha$ in a path from
$s$ to $v$ whose cost is at most $c$.
%
Also, let $\latency(\gamma,\beta,w \leadsto v)$ be the minimum
possible latency of path from $w$ to $v$ such that the bandwidth of
all edges in the path is at least $b(\gamma,\beta)$.  That is,
\[ 
\latency(\gamma,\beta,w \leadsto v)
= \min_{f: b(\gamma,\beta) \leq \min_{e \in \bar{f}(\gamma,\beta)} b(e)}
    \sum_{e \in \bar{f}(\gamma,\beta)} \ell((\gamma,\beta),e)
~.
\]
Observe that $\latency(\gamma,\beta,w \leadsto v)$ can be computed
using a \textsc{Shortest Path} algorithm, were we consider only edges
whose bandwidth cap is at least $b(\gamma,\beta)$.
%
Now we are ready for the computation of $\latency(\alpha,v,c)$:
\[
\latency(\alpha,v,c) =
\begin{cases}
\infty  &  c < 0,\\
0       &
\begin{array}{c}
\alpha = \sigma, v = s, \\
\text{and } c \geq 0,
\end{array} \\
\displaystyle
\min_{
  \substack{\beta \prec \alpha, (\gamma,\beta) \in \calE, \\
           w \prec v, (u,w) \in E, \\
         (u,v) \Mapsto (\gamma,\beta)}
     }
     \set{\begin{array}{l}
          \latency(\gamma,\beta,w \leadsto v) + \\
          \latency(\gamma,u,c - \cost_v(\beta \leadsto \alpha))
          \end{array}
     }
  & \text{otherwise}.
\end{cases}
\]
The minimum latency of a placement of a path to $\alpha$ in a path to
$v$ whose cost is at most $c$ can be divided into two values: the
latency caused by the placement of a path to $\gamma$, where
$(\gamma,\beta) \in \calE$, and $\beta \prec \alpha$, with cost $c -
\cost_v(\beta \leadsto \alpha)$, and the latency caused by placing a
virtual edge $(\gamma,\beta)$ on a physical path from $w$ to $v$,
where $w \prec v$ and $(u,w) \in E$.
%
The optimal value is $\min \set{c : \latency(\tau,t,c) \leq L}$.

Since the computation of $\latency(\gamma,\beta,w \leadsto v)$ can be
done efficiently, the computation for a triple $(\alpha,v,c)$ can be
done in polynomial time.  Hence, the total running time is polynomial.
Finally, as in the previous algorithms, one may use the algorithm to
compute a corresponding placement
%that achieves this value
using standard techniques.

%%%%%

\subsection{FPTAS for General Costs}
\label{sub:fptas}

In this section we present an FPTAS for \scp in DAGs for general
costs, namely without the assumption that costs are polynomially
bounded.  Our algorithm is similar to the FPTAS for the
\textsc{Minimum Knapsack} problem.

Let $f$ be an optimal placement, and let $c_{\max}$ be the maximum
cost of a VNF placement by $f$, i.e., $c_{\max} = \max_{f(\alpha) =
  u} c(\alpha,v)$.
%
Given a constant $\eps>0$, we define
\[
c'(\alpha, v) \eqdf
\begin{cases}
\ceil{\frac{c(\alpha, v)}{c_{\max}} \cdot \frac{n}{\eps}}
       & c(\alpha, v) \leq c_{\max}, \\
\infty & c(\alpha, v) > c_{\max}.
\end{cases}
\]
Let $f'$ be an optimal placement with respect to $c'$.

\begin{lemma}
\label{lemma:guess}
$c(f') \leq (1+\eps)c(f)$.
\end{lemma}
\begin{proof}
We have that
\begin{align*}
c(f')
%&
=    \sum_{\alpha} c(\alpha,f'(\alpha)) %\\
& \leq \frac{\eps c_{\max}}{n} \cdot \sum_{\alpha} c'(\alpha,f'(\alpha)) \\
& \leq \frac{\eps c_{\max}}{n} \cdot \sum_{\alpha} c'(\alpha,f(\alpha)) \\
& \leq \frac{\eps c_{\max}}{n} \cdot
     \sum_{\alpha}
       \paren{ c(\alpha,f(\alpha)) \cdot \frac{\eps c_{\max}}{n} + 1} \\
%& =    c(f) + \frac{\eps c_{\max}}{n} \cdot n \\
& =    c(f) + \eps c_{\max} \\
& \leq (1+\eps) c(f)
~,
\end{align*}
where the second inequality is due to the optimality of $f'$ with
respect to $c'$.
%
\qed
\end{proof}

This leads to the following result:

\begin{theorem}
\label{thm:fptas}
There exists an FPTAS for \scp in DAGs.
\end{theorem}
\begin{proof}
Given $\eps>0$, run the dynamic programming algorithm from
Section~\ref{sub:DPL} $\abs{\calV} \cdot \abs{V}$ times, once for each
possible value of $c_{\max}$, and choose the best placement.
According to Lemma~\ref{lemma:guess} the best placement is
$(1+\eps)$-approximate.  The running time is polynomial in the input
size and in $1/\eps$.
%
\qed
\end{proof}


%%%%%%%%%%%%%%%%%%%%%%%%%%%%%%%%%%%%%%%%%%%%%%%%%%%%%%%%%%%%%%%%%%%%%

\section{General Networks}
\label{sec:general}

In this section we consider \scp when the physical network is an
undirected graph.  
Recall that even the feasibility version of \scp is NP-hard in general networks,
and therefore we focus on a special case.
%
A vertex $v \in V$ is called \emph{neighborly} if it has more than two
neighbors.
%
First, we assume that there exists an optimal placement whose physical
path consists of at most $k$ neighborly vertices.  For example, this
can be assumed if all simple paths from $s$ to $t$ contain at most $k$
neighborly vertices.  In this case we present a randomized algorithm
that computes an $(1+\eps)$-approximate placement with high
probability whose running time is $O(k! \cdot \text{poly}(n))$.
%
Our second algorithm works under the stronger assumption that there
are $k$ neighborly vertices in the network.  In this case, we present
a deterministic algorithm whose running time is $O(k! \cdot
\text{poly}(n))$.

%%%%%

Our randomized algorithm consists of $k!$ iterations of the
following two phases: an orientation phase that applies a random
orientation to the physical network, and an execution of the FPTAS for
DAGs given in Theorem~\ref{thm:fptas}.  The algorithm finds a
$(1-\eps)$-approximate placement with probability $(1-1/e)$, and the
running time of the algorithm is $O((n+t(n))k!)$, where $t(n)$ is the
time it takes to compute a placement when the physical network is a
DAG.
%
As usual, one may amplify the probability of success using repetition.

The orientation phase is done as follows: 
let $N$ be the set of neighborly vertices, 
and let $\pi:N \to \set{1,\ldots,\abs{N}}$ be a random permutation.  
We direct the edges according to $\pi$.  
Observe that each edge $e \in E$ is found on a simple path between two
neighborly nodes $v_e$ and $v'_e$, in which all internal vertices are
in $V \subseteq N$.  
The edge $e$ is directed towards $v_e$, if $\pi(v_e) > \pi(v'_e)$, 
and otherwise it is directed towards $v'_e$.
%
Observe that when this process terminates we get a DAG, denoted by
$G_\pi$, where there is a topological order than is consistent which
$\pi$.  Figure~\ref{fig:orientation} depicts the orientation stage.
%
A na\"ive implementation of this phase would run in $O(|V||E|)$ time.

\begin{figure}[t]
\centering
\scalebox{0.9}{
  \subfloat[Original Network.]{%
\begin{tikzpicture}[every node/.style={default node}]
\node(6) at(0, 0){6};
\node(2) at(1, 1){2};
\node(1) at(3, 2){1};
\node(3) at(5, 1){3};
\node(5) at(4,-1){5};
\node(4) at(2,-3){4};

\def\pat{\draw[path, -]}
\pat[blue, dashed, thick] (1) -- (2.north east);
\pat (2) -- (3);
\pat (3) -- (5);
\pat (2.south east) -- (5);
\pat (2) -- (6);
\pat (4) -- (6);
\pat (4) -- (5);
\pat (1) -- (3);
\pat[blue, dashed, thick] (2) -- (4);
\pat (1.north) to[bend right] (6.north);
\pat (3.south east) to[bend left] (4.east);
\end{tikzpicture}
}

\hspace{50pt}

\subfloat[Orientated network.]{%
\begin{tikzpicture}[every node/.style={default node}]
\node(6) at(6, 0){6};
\node(2) at(7, 1){2};
\node(1) at(9, 2){1};
\node(3) at(11, 1){3};
\node(5) at(10,-1){5};
\node(4) at(8,-3){4};

\def\dpat{\draw[path]}
\dpat[blue, dashed, thick] (1) -- (2.north east);
\dpat (2) -- (3);
\dpat (3) -- (5);
\dpat (2.south east) -- (5);
\dpat (2) -- (6);
\dpat (4) -- (6);
\dpat (4) -- (5);
\dpat (1) -- (3);
\dpat[blue, dashed, thick] (2) -- (4);
\dpat (1.north) to[bend right] (6.north);
\dpat (3.south east) to[bend left] (4.east);
\end{tikzpicture}
}

}
\caption{Orientation phase: on the left is the physical network, where
  only the heavy nodes are drawn.  The numbers represent the
  permutation and the dashed path represents a path of an optimal
  placement.  The orientated physical network is on the right.  It is
  enough that the internal ordering of the nodes on the dashed path is
  "correct" to ensure that the survival of the optimal placement. }
\label{fig:orientation}
\end{figure}

Upon completing the orientation, we use our previous algorithm to find
a placement.  We repeat this process $k!$ times, each time with a new
independent random permutation, and keep the best embedding so far.

\begin{theorem}
There exists an algorithm, that given an \scp instance with an optimal
solution that contains at most $k$ neighborly vertices, finds a
$(1+\eps)$-approximate placement with high probability, whose
running time is $O(k! \cdot \text{\rm poly}(n))$.
\end{theorem}
\begin{proof}
Consider an optimal placement with at most $k$ neighborly vertices,
and let $\pi'$ be the permutation it induces on the neighborly
vertices in the physical path.
%
If the random permutation $\pi$ orders the neighborly vertices
correctly, i.e., if $\pi(v) = \pi'(v)$, for every $v \in N$, then the
optimal placement is feasible with respect to $G_\pi$.  Hence, the
FPTAS will find a $(1+\eps)$-approximate placement in $G_\pi$.  This
placement is feasible, and therefore $(1+\eps)$-approximate placement,
with respect to $G$.
%
The permutation $\pi$ agrees with $\pi'$ with probability $1/k!$, and
the probability that $\pi$ agrees with $\pi'$ in $k!$ iterations is $1
- (1 - 1/k!)^{k!} \geq 1 - e^{-1}$.  One may amplify the success
probability to $1 - \inv{n}$ using $\log n$ repetitions.
%
\qed
\end{proof}

In the case where $\abs{N} = k$ we can simply examine the $k!$
permutations of the neighborly vertices.  For each possible
permutation the algorithm applies the graph orientation procedure as
described above, and then executes the FPTAS for DAGs given in
Section~\ref{sub:fptas}.  
Thus the running time of the algorithm is $O(k! \cdot \text{\rm poly}(n))$.
%
We note that permutation enumeration for $k$ numbers takes $O(k!)$
time (see e.g.,~\cite{Even73,Sedgewick77})

\begin{theorem}
There exists an $O(k! \cdot \text{\rm poly}(n))$-time algorithm, that
given an \scp instance with $k$ neighborly vertices, finds a
$(1+\eps)$-approximate placement.
\end{theorem}


%%%%%%%%%%%%%%%%%%%%%%%%%%%%%%%%%%%%%%%%%%%%%%%%%%%%%%%%%%%%%%%%%

\section{FPT Algorithms for General Networks}
We now propose an FPT algorithm for general networks that is similar to the randomized 
algorithm from Section~\ref{sec:general}.
Our FPT algorithm runs in $O(k! \cdot (A + O))$ where $k$ is the number of neighborly
vertices in the graph, $A$ is the time required to find a solution on a DAG,
and $O$ is the time required to orient the graph given a permutation on the neighborly
vertices, as discussed above.
The algorithm executes the FPTAS for DAGs given in Theorem~\ref{thm:fptas}
once for any orientation results from any of the $k!$ possible permutations and
returns the best solution.
The following theorem follows immediately.

\begin{theorem}
There is an FPT algorithm that finds a $(1+\eps)$-approximate placement after $k!$ iterations 
\end{theorem} 

\subsection*{Discussion About the Orientation Phase}
We now discuss the orientation phase and propose a different method to apply 
an orientation to an undirected graph.
While the proposed method does not surpass the trivial orientation in the worse case, 
in certain cases, the new method yields an exponential improvement over the trivial one.

The orientation works as follow:
given an undirected graph $G = (V, E)$ we first find its block tree 
(where every vertex in the tree is a maximal subgraph of $G$ without a cut-vertex),
then, on the unique  path from $s$ to $t$ we internally orient each block arbitrarily to
produce a DAG.
Note that if $B_1, \ldots, B_k$ are the blocks of $G$ on the path from $s$ to $t$ then
the first and last vertices w.r.t to each block are known and only the order
of the internal neighborly vertices in each block is needed to be determined,
thus, we can find all relevant orientations by repeating the process above
$\Pi_{i = 1}^k n(B_i)$ times, where $n(B_i)$ is the number of internal neighborly vertices
in $B_i$.

\begin{theorem}
There is an FPT algorithm that finds a $(1+\eps)$-approximate placement after
$\Pi_{i = 1}^k n(B_i)$ iterations.
\end{theorem}

Figure~\ref{fig:or2} shows that in some cases this approach is faster than the trivial one.
We also note that if the given graph is a cactus graph then using this approach result in 
a polynomial algorithm.
This special case of cactus graph was already considered in \cite{Stefan} and was also found
as a special case for which a hard problem becomes tractable. 

\begin{theorem}
There is a polynomial algorithm that finds a $(1+\eps)$-approximate placement 
for cactus graph.
\end{theorem}


\begin{figure}
\begin{center}
\begin{tikzpicture}[every node/.style={default node}, -]

\node(s) at (0,0) {s}; 

\begin{scope}[yshift=1cm, xshift=1cm]
\foreach[count=\i] \x \y in {%
0/0,0/-2,2/-1,4/1,3/-2}{
	\node(\i) at (\x,\y){\i};
}
\end{scope}

\begin{scope}[yshift=-5cm]
\foreach[count=\i from 6] \x \y in {%
0/0,0/3,1/0,2/2}{
	\node(\i) at (\x,\y){\i};
}
\end{scope}

\begin{scope}[yshift=-5cm, xshift=4cm]
\foreach[count=\i from 10] \x \y in {%
-1/0,0/1,1/-1,1/2}{
	\node(\i) at (\x,\y){\i};
}
\end{scope}

\begin{scope}[yshift=-5cm, xshift=-3cm]
\foreach[count=\i from 14] \x \y in {%
0/0,1/2,2/1,2/3,0/4}{
	\node(\i) at (\x,\y){\i};
}
\end{scope}


\node(t) at (4,-7) {t};

\foreach \u \v in{%
s/1,s/18%
,1/2,1/3,1/4%
,2/3,2/5%
,3/4,3/5%
,4/5%
,5/9%
%
,6/7,6/8,6/9%
,7/9%
,8/9%
,8/10%
%
,10/11,10/12%
,11/13%
,12/13%
%
,14/16,14/15,14/18%
,15/16,15/17,15/18%
,16/17%
,17/18%
%
,12/t%
}{
	\draw (\u) -- (\v);
}

\draw[dashed, green]
(1.north west) to[bend left] 
(4.north east) to[bend left]
(5.south east) to[bend left]
(2.south west) to[bend left]
(1.north west)
;

\draw[dashed, green]
(7.north west) to[bend left] 
(9.north east) to[bend left]
(8.south east) to[bend left]
(6.south west) to[bend left]
(7.north west)
;

\draw[dashed, green]
(10.west) to[bend left] 
(13.north) to[bend left]
(12.south) to[bend left]
(10.west) 
;
\end{tikzpicture}

\end{center}
\caption{
A undirected graph with its block marked in a dashed, green lines.
In this instance, using the trivial approach the orientation phase takes 16! iterations,
while using the second approach we only need 3! iterations.
\label{fig:or2}
}
\end{figure}   

%%%%%%%%%%%%%%%%%%%%%%%%%%%%%%%%%%%%%%%%%%%%%%%%%%%%%%%%%%%%%%%%%

\section{Fault Tolerance}
In this section we consider the problem of a service chain placement when there 
is also demand for fault tolerance.
Formally, given a physical network, a virtual specification and a parameter 
$\phi \in \mathbb{N}$, a $\phi$-robust placement is a collection of $\phi$ placements,
$f_1,\ldots,f_\phi$,
that correspond to $\phi$ internal vertex disjoint physical paths. 
We are looking for a $\phi$-robust placement that minimize the total cost:
$$
\sum_{i = 1}^\phi \sum_j c(\alpha_j,f_i(\alpha_j))
$$
In this section we discuss how to efficiently find an optimal 
(assuming integral polynomial bounded costs) 2-robust placement.
The algorithm presented here, however, can be generalized to find an optimal $\phi$-robust
placement for any $\phi$

\subsection{Failure of the Greedy Algorithm}
Here we give a simple example why the greedy algorithm fails badly.
By greedy we mean applying the previous algorithm once to find an optimal placement, 
then removing all the vertices from the resulted path and applying the same algorithm again
to find the second placement.
Consider the physical graph as described in Figure~\ref{fig:greedy-bad} and a virtual path of
two virtual functions $\alpha$ and $\beta$.
Assume latencies, bandwidth, and processing constraints are all negligible 
and assume the following cost function 
$c(1, \alpha) = c(2, \alpha) = 1-\varepsilon, c(1, \beta) = c(2, \alpha) = 1$.
The greedy algorithm is unable to find a 2-robust placement in this case - after the 
first iteration there is no path from $s$ to $t$ - while, clearly there is a 2-robust
placement that costs $4 - 2 \epsilon$.   
\begin{figure}
\caption{
\label{fig:greedy-bad}
}
\begin{center}
\begin{tikzpicture}[every node/.style={default node}]
\foreach \x \y \n in {0/0/s, -1/-1/1, 1/-1/2, 0/-2/t}{
	\node(\n) at(\x,\y) {\n};
}
\foreach \u \v in {%
	s/1%
	,s/2%
	,1/2%
	,1/t%
	,2/t%
}{
	\draw (\u) -- (\v);
}
\end{tikzpicture}
\end{center}
\end{figure}

  
\subsection{Fault Tolerance on Layered Graph}
We first assume that the input graph is a directed layered graph, i.e.
$G = (\bigcup_{i = 1}^l L_i, A)$ where 
$A \subseteq \bigcup_{i = 1}^{l - 1} L_i \times L_{i+1}$.
We assume that the number of vertices in each layer is at least two, or otherwise a 2-robust 
placement does not exist.
The algorithm uses dynamic programming.
We find the optimal 2-robust placement for every  
combination of two physical vertices from the same layer, 
and every combination of two virtual functions.
Formally, given the input $(G = (V, E), \calG = (\calV, \calE), p, b, s, t, c)$
We define $ft(v_1, v_2, \alpha_1, \alpha_2)$ to be the optimal 2-robust placement of 2 chains
that start at $\sigma$ and end at $\alpha_1, \alpha_2$ respectively into 
(internal vertex disjoint) physical paths that start at $s$ and end at $v_1,v_2$ respectively.
One can verified that the following recursive formula holds:
$$
ft(v_1, v_2, \alpha_1, \alpha_2, \gamma_1\beta_1, \gamma_2\beta_2) = \min
\begin{cases}
nn(v_1, v_2, \alpha_1, \alpha_2, \gamma_1\beta_1, \gamma_2\beta_2)
\\
ny(v_1, v_2, \alpha_1, \alpha_2, \gamma_1\beta_1, \gamma_2\beta_2)
\\
yn(v_1, v_2, \alpha_1, \alpha_2, \gamma_1\beta_1, \gamma_2\beta_2)
\\
yy(v_1, v_2, \alpha_1, \alpha_2, \gamma_1\beta_1, \gamma_2\beta_2) 
\end{cases}
$$

where
$$
\begin{array}{l}
nn(v_1, v_2, \alpha_1, \alpha_2, \gamma_1\beta_1, \gamma_2\beta_2) = 
\displaystyle\smashoperator{\min_{\substack{\\ u_1, u_2: \\ b(u_1, v_1) \geq b(\gamma_1, \beta_1) \\ b(u_2, v_2) \geq b(\gamma_2, \beta_2)}}}
ft(u_1, u_2, \alpha_1, \alpha_2, \gamma_1\beta_1, \gamma_2\beta_2)
\\
ny(v_1, v_2, \alpha_1, \alpha_2, \gamma_1\beta_1, \gamma_2\beta_2) = 
\displaystyle\smashoperator{\min_{
	\substack{
	u_1, u_2: 
	\\ 
	b(u_1, v_1) \geq b(\gamma_1, \beta_1) 
	\\ 
	b(u_2, v_2) \geq b(\gamma_2, \beta_2)
	\\
	\gamma'_2\beta'_2 \in \calE:
	\\
	\beta'_2 \prec \gamma_2 
	}}}
ft(u_1, u_2, \alpha_1, \gamma_2, \gamma_1\beta_1, \gamma'_2\beta'_2)
\\
yn(v_1, v_2, \alpha_1, \alpha_2, \gamma_1\beta_1, \gamma_2\beta_2) = 
\displaystyle\smashoperator{\min_{
	\substack{
	u_1, u_2: 
	\\ 
	b(u_1, v_1) \geq b(\gamma_1, \beta_1) 
	\\ 
	b(u_2, v_2) \geq b(\gamma_2, \beta_2)
	\\
	\gamma'_1\beta'_1 \in \calE:
	\\
	\beta'_1 \prec \gamma_1 
	}}}
ft(u_1, u_2, \gamma_1, \alpha_2, \gamma'_1\beta'_1, \gamma_2\beta_2)
\\
yy(v_1, v_2, \alpha_1, \alpha_2, \gamma_1\beta_1, \gamma_2\beta_2) = 
\displaystyle\smashoperator{\min_{
	\substack{
	u_1, u_2: 
	\\ 
	b(u_1, v_1) \geq b(\gamma_1, \beta_1) 
	\\ 
	b(u_2, v_2) \geq b(\gamma_2, \beta_2)
	\\
	\gamma'_1\beta'_1, \gamma'_2\beta'_2 \in \calE:
	\\
	\beta'_1 \prec \gamma_1, \beta'_2 \prec \gamma_2 
	}}}
ft(u_1, u_2, \gamma_1, \gamma_2, \gamma'_1\beta'_1, \gamma'_2\beta'_2)
\end{array}
$$

\begin{theorem}
There is an algorithm that, on a layered graph, finds an optimal 2-robust placement in 
polynomial time. 
\end{theorem}

\begin{proof}
One can verify, that using dynamic programming, the above recursive can be solved in polynomial time,
using backtracking we can also find the optimal solution.
\end{proof}

We note that the above algorithm does not take into account the latency constraint.
It is possible, however, given a latency constraint, $L$, to modify the algorithm to return
an optimal solution such that the total latency of the two placements is no more than $2L$.
Such a solution is guaranteed to be optimal in terms of costs.
It is also easy to see that in such solution at least one of the placements has lower latency 
than $L$ and the other one has latency that is at most $2L$. 

\subsection{Layered Graph Transformation}
Given a directed acyclic graph $G = (V, A)$, 
and let $\pi:V \to [n]$ be an arbitrary topological sorting of $G$ 
we transform it into a layered graph $G'$ by replacing every edge $(u, v)$ with a directed path of 
length (number of arcs) $\pi(v) - \pi(u)$ from $u$ to $v$.
We set the processing capacity of each internal vertex on this path to be zero,
and the bandwidth of the arcs along the path to be $b(u,v)$.
It is easy to see that the resulting graph is indeed a layered graph and that every placement
on $G$ corresponds to a placement on $G'$ with the same cost.
Figure~\ref{fig:layer-transform} depicts this process.
\begin{figure}
\captionsetup{singlelinecheck=off}
\caption[layered]{foo
\label{fig:layer-transform}
\begin{enumerate}
\item A directed acyclic graph that is not a layered graph
  \item A topological sorting of the above graph
  \item A layered graph obtained by replacing crossing arcs with simple paths 
\end{enumerate}}
\begin{center}
\begin{tikzpicture}[every node/.style={default node}]

\foreach[count=\i] \x \y in {
	0/0
	,2/1
	,1/-2
	,3/-1
	,4/0
	,5/-2
	,6/1
}{
	\node(\i) at(\x,\y) {\i};
}

\foreach \u \v in {%
	1/2%
	,1/3%
	,2/3%
	,2/4%
	,2/7%
	,3/4%
	,3/6%
	,4/5%
	,4/6%
	,5/6%
	,6/7%
}{
	\draw (\u) -- (\v);
}

\begin{scope}[yshift=-5cm, x=1.5cm]
\foreach[count=\i] \x \y in {
	0/0
	,1/0
	,2/0
	,3/0
	,4/0
	,5/0
	,6/0
}{
	\node(\i) at(\x,\y) {\i};
}

\foreach \u \v in {%
	1/3%
	,3/6%
}{
	\draw (\u) to[bend left] (\v);
}
\foreach \u \v in {%
	2/4%
	,2/7%
	,4/6%
}{
	\draw (\u) to[bend right] (\v);
}
\foreach \u \v in {%
	1/2%
	,2/3%
	,3/4%
	,4/5%
	,5/6%
	,6/7%
}{
	\draw (\u) to[] (\v);
}
\end{scope}

\begin{scope}[yshift=-10cm, x=1.5cm]
\foreach[count=\i] \x \y in {
	0/0
	,1/0
	,2/0
	,3/0
	,4/0
	,5/0
	,6/0
}{
	\node(\i) at(\x,\y) {\i};
}

\foreach[count=\i] \x \y in {
	1/1
	,2/-1,2/-2
	,3/1,3/-2
	,4/1,4/-1,4/-2
	,5/-2
}{
	\node(d\i) at(\x,\y) {d\i};
}

\foreach \u \v in {%
	1/2%
	,1/d1,d1/3%
	,2/3%
	,2/d2,d2/4%
	,2/d3,d3/d5,d5/d8,d8/d9,d9/7%
	,3/4%
	,3/d4, d4/d6, d6/6%
	,4/5%
	,4/d7, d7/6%
	,5/6%
	,6/7%
}{
	\draw (\u) to[] (\v);
}
\end{scope}
\end{tikzpicture}
\end{center}
\end{figure}
 



%%%%%%%%%%%%%%%%%%%%%%%%%%%%%%%%%%%%%%%%%%%%%%%%%%%%%%%%%%%%%%%%%%%%%

%\section{Extensions}

%Costs by sets of VNFS

%Fault tolerant placement: two service chains?

%FPT algorithm where the number of servers whose degree is larger than
%two is a parameter?

%%%%%%%%%%%%%%%%%%%%%%%%%%%%%%%%%%%%%%%%%%%%%%%%%%%%%%%%%%%%%%%%%%%%%

\subsection*{Acknowledgement}

We thank Reuven Bar-Yehuda for helpful discussions.


%%%%%%%%%%%%%%%%%%%%%%%%%%%%%%%%%%%%%%%%%%%%%%%%%%%%%%%%%%%%%%%%%%%%%

\bibliographystyle{abbrv}
\bibliography{chain}

\end{document}
