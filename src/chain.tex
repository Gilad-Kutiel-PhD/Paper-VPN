%\nonstopmode
%\documentclass[11pt]{article}
\documentclass[runningheads]{llncs}

\usepackage{amsmath,amssymb,mathtools}
%\usepackage{amsthm}
%\usepackage{fullpage}
\usepackage{xspace}
\usepackage{color,soul}
\usepackage{paralist}

% \usepackage[linesnumbered, ruled]{algorithm2e}

\usepackage[final]{microtype}
\usepackage[final]{hyperref}
\usepackage{subcaption}

% DEBUG
%\usepackage{lipsum}
\usepackage{todonotes}
%\usepackage{lineno}
%\linenumbers

% EXTRA
%\usepackage{authblk}

\usepackage{tikz}
\usetikzlibrary{
	arrows,
	arrows.meta,
	calc,
	patterns,
	positioning,
	shapes,
	decorations.pathmorphing,
	decorations.markings,
}

% \usetikzlibrary{graphs}
\usetikzlibrary{graphdrawing}
\usegdlibrary{trees,force,layered}

%DEFAULT STYLE
\tikzset{default style/.style={
	>=Stealth, 
	on grid, 
	auto, 
	thick,
}}

% NODE
\tikzset{default node/.style={
	draw, 
	circle,
	inner sep=0mm,
	minimum size=5mm,
	very thick,
	font=\small,
	black!70,
}}

\tikzset{clear node/.style={
	draw=none, 
	inner sep=0mm,
	minimum size=0mm,
}}

% LABELS
\tikzset{label/.style={
	draw=none,
	sloped,
}}

\tikzset{label above/.style={
	label,
	midway,
	above=-.2mm,
}}

\tikzset{label below/.style={
	label,
	midway,
	below=-.2mm,
}}


\sloppy

%%%%%%%%%%%%%%%%%%%%%%%%%%%%%%%%%%%%%%%%%%%%%%%%%%%%%%%%%%%%%%%%%%%%%

%% handy commands

\newcommand{\eqdf}{\stackrel{\scriptscriptstyle \triangle}{=}}
\newcommand{\argmin}{\ensuremath{\mbox{argmin}}}
\newcommand{\argmax}{\ensuremath{\mbox{argmax}}}
\newcommand{\set}[1]{\left\{ #1 \right\}}
\newcommand{\paren}[1]{\left( #1 \right)}
\newcommand{\bit}{\set{0,1}}
\newcommand{\inv}[1]{\frac{1}{#1}}
\newcommand{\abs}[1]{\left| #1 \right|}
\newcommand{\ceil}[1]{\left\lceil {#1} \right\rceil}
\newcommand{\floor}[1]{\left\lfloor {#1} \right\rfloor}
\newcommand{\half}{\frac{1}{2}}
\newcommand{\threehalves}{\frac{3}{2}}

\newcommand{\naturals}{\mathbb{N}}
\newcommand{\rationals}{\mathbb{Q}}
\newcommand{\reals}{\mathbb{R}}
\newcommand{\integers}{\mathbb{Z}}

\newcommand{\eps}{\varepsilon}

%%%%%%%%%%%%%%%%%%%%%%%%%%%%%%%%%%%%%%%%%%%%%%%%%%%%%%%%%%%%%%%%%%%%%

\newcommand{\scp}{\textsc{SCP}\xspace}
\newcommand{\scplong}{\textsc{Service Chain Placement}\xspace}

\newcommand{\calE}{\mathcal{E}}
\newcommand{\calG}{\mathcal{G}}
\newcommand{\calV}{\mathcal{V}}

\newcommand{\dror}[1]{\sethlcolor{yellow}\hl{Dror: #1}}


%%%%%%%%%%%%%%%%%%%%%%%%%%%%%%%%%%%%%%%%%%%%%%%%%%%%%%%%%%%%%%%%%%%%%

\title{\bf Service Chain Placement in SDNs%
\thanks{Research supported in part by Network Programming (Neptune)
  Consortium, Israel.}
}

\titlerunning{Service Chain Placement in SDNs}


%\author[1]{Gilad Kutiel%
%\thanks{E-mail: \texttt{gkutiel@cs.technion.ac.il}}}

%\author[2]{Dror Rawitz%
%\thanks{Partially supported by the Israel Science Foundation
%  (grant no.~497/14).  E-mail: \texttt{dror.rawitz@biu.ac.il}}}

%\affil[1]{Department of Computer Science, Technion, Haifa, Israel}
%\affil[2]{Faculty of Engineering, Bar Ilan University, Ramt-Gan, Israel}

\author{%
Gilad Kutiel
\and
Dror Rawitz%
\thanks{Partially supported by the Israel Science Foundation
  (grant no.~497/14).  }
}

\institute{%
Department of Computer Science, Technion, Haifa 32000, Israel. \\
\email{gkutiel@cs.technion.ac.il}
\and
Faculty of Engineering, Bar Ilan University, Ramt-Gan 52900, Israel.\\
\email{dror.rawitz@biu.ac.il}
}

%%%%%%%%%%%%%%%%%%%%%%%%%%%%%%%%%%%%%%%%%%%%%%%%%%%%%%%%%%%%%%%%%%%%%

\begin{document}
\maketitle

\begin{abstract}
We study the allocation problem of a \emph{service chain} in a
software-defined network that supports network function
virtualization.
%
Given a network that contains servers with limited processing power
and of links with limited bandwidth, a service chain is a sequence of
virtual network functions (VNFs) that service a certain flow request
in the network.
%
The allocation of a service chain consists of routing and VNF
placement.  That is, each VNF from the seqeunce is placed in a server
along a path, and it is feasible if each server can handle the VNFs
that are assigned to it, and if each link on the path can carry the
flow that is assigned to it.
%
A request for service is composed of a source and a destination in the
network, an upper bound on the total latency, and a specification of
all service chains that are considered valid for this request.
%
Each pair of server and VNF are associated with a cost for placing the
network function in the server, and given a request, the goal is to
find a valid service chain of minimum total cost or to identify that
a valid service chain does not exist.

We show that even the feasibility problem is NP-hard in general
di-graphs.  Hence we first focus on directed acyclic graphs (DAGs).
We show that the problem is still NP-hard in DAGs and present an
FPTAS.
%
Based on out PTAS, we also provide a randomized algorithm for
instances in which the service chain passes through a bounded number
of servers whose degree is larger than two.

?????

Fault tolerant placement: two service chains?

FPT algorithm where the number of servers whose degree is larger than
two is a parameter?
\end{abstract}

%%%%%%%%%%%%%%%%%%%%%%%%%%%%%%%%%%%%%%%%%%%%%%%%%%%%%%%%%%%%%%%%%%%%%

\section{Introduction}


Computer communication networks are in constant need of expansion to
cope with the ever growing traffic.  As networks grow, management and
maintenance become more and more complicated.
%
Current developments aim to improve the utilization of network
resources include the detachment of network applications from network
infrastructure and the transition from network planning to network
programming.

One aspect of network programming is to manage resources from a
central point of view, namely to make decisions based on availability,
network status, required quality of service, and the identity of the
client.  Hence, a central issue is a central agent that is able to
received reports from network components and client requests, and as a
result can alter the allocation of resources in the networks.  This
approach is called Software Defined Networking (SDN), where there is a
separation between routing and management (control plane) and the
underlying routers and switches that forward traffic (data plane) (see
Kreutz et al.~\cite{KRVRAU15}).

A complimentary approach is Network Function Virtualization
(NFV)~\cite{NFV12}.  Instead of relying on special purpose machines,
network applications become virtual network functions (VNF) that are
executed on generic machines and can be placed in various locations in
the network.  Virtualization increases the flexibility of resource
allocation and thus the utilization of the network resources.
%
Internet Service Providers (ISPs) that provide services to clients
benefit from NFV, since it helps to better utilize their physical
network.  In addition, when network services are virtualized, an ISP
may support \emph{service chaining}~\cite{ServiceChaining15}, namely a
compound service that consists of a sequence of VNFs.  Furthermore,
one may offer a compound service a service that may be obtained using
one of several sequences of VNFs.

In this paper we study the allocation problem of a compund request for
a service chain in a software-defined network that supports NFV.
Given a (physical) network that contains servers with limited
processing power and of links with limited bandwidth, a service chain
is a sequence of virtual network functions (VNFs) that service a
certain flow request in the network.
%
The allocation of a service chain consists of routing and VNF
placement.  That is, each VNF from the seqeunce is placed in a server
along a path, and it is feasible if each server can handle the VNFs
that are assigned to it, and if each link on the path can carry the
flow that is assigned to it.
%
A request for service is composed of a source and a destination in the
network, an upper bound on the total latency, and a specification of
all service chains that are considered valid for this request.
%
Each pair of server and VNF are associated with a cost for placing the
network function in the server.  This cost measures compatibility of a
VNF to a server (infinite costs means ``incompatible'').  Given a
request, the goal is to find a valid service chain of minimum total
cost or to identify that a valid service chain does not exist.


\dror{Explain one shot approach}

\paragraph*{\bf Related work.}
%

Cohen et al.~\cite{CLNR15} considered VNF placement.  In their model
the input is an undirected graph with a metric distance function on
the edges.  clients are located in various nodes, and each client is
interested in a subset of VNFs.  For each node $v$ and VNF $\alpha$,
there is a setup cost associated with placing a copy of $\alpha$ at
$v$ (multiple copies are allowed), and there is a load that is induced
on $v$ for placing a copy of $\alpha$.  Furthermore, each node has a
limited capacity.  Also, each copy of a VNF can handle a limited
amount of clients A solution is the assignment of each client to a
subset of nodes, each corresponding to one of its required VNFs.  The
cost of a solution is the total setup costs plus the sum of distances
between the clients and the node from which they get service.
%
Cohen et al.~\cite{CLNR15} gave bi-criteria approximation algorithms
for various versions of the problem, namely algorithms that compute
constant factor approximations that violate the load constraints by a
constant factor.
%
It is important to note that in this problem does not consider
routing, and also it assumes unordered VNF subsets.

Even, Medina, and Patt-Shamir~\cite{EMP16} studied online path
computation and VNF placement.  They considered service chain requests
that arrive in an online manner.  Each request is a a flow with a
high-level specification of routing and VNF requirements.  Each VNF
can be performed by a specified subset of servers in the system.  Upon
arrival of a request, the algorithm either rejects the request or
accepts it and with a specific routing and VNF assignment, under given
server capacity constraints.  Each request has a benefit that is
gained if it is accepted, and the goal is to maximize the total
benefit.
%
Even, Medina, and Patt-Shamir~\cite{EMP16} proposed an algorithm that
copes with requests with unknown duration without preemption by using
a third response, which is refer to as “stand by”, whose competitive
ratio is $O(\log (knb_{\max})$, where $n$ is the number of nodes,
$b_{\max}$ is an upper bound on a request benefit per time unit, and
$k$ is the maximum number of VNFs of any service chain.  This
performance guarantee holds for the case where the processing of any
processing request in any possible service chain is never more than
$O(1/(k \log (nk)))$ fraction of the capacity of any network
component.

Even, Rost, and Schmid~\cite{ERS16}

\dror{finish}

\paragraph*{\bf Our results.}
%
We show that even the feasibility problem is NP-hard in general
di-graphs.  Hence we first focus on directed acyclic graphs (DAGs).
We show that the problem is still NP-hard in DAGs and present an
FPTAS.
%
Based on out PTAS, we also provide a randomized algorithm for
instances in which the service chain passes through a bounded number
of servers whose degree is larger than two.

?????

\bigskip



%%%%%%%%%%%%%%%%%%%%%%%%%%%%%%%%%%%%%%%%%%%%%%%%%%%%%%%%%%%%%%%%%%%%%

\section{Preliminaries}


\paragraph*{\bf Model.}
%
The \scplong (\scp) input is composed of three components, a physical
network, a virtual specification, and placement costs:
\begin{description}
\item[Physical network:]
%
  The physical network is a digraph $G = (V,E)$.  Each node $v \in V$
  has a non-negative processing capacity $p(v)$, and each directed
  edge $e \in A$ has a non-negative bandwidth cap $b(e)$.

\item[Virtual specification:]
%
  The description of a request for a service chain consists of a
  physical source $s \in V$, a physical destination $t \in V$, and a
  directed acyclic graph (DAG) $\calG = (\calV,\calE)$.
%
  Without loss of generality, we assume that $p(s) = p(t) = 0$.  
%
  The DAG $\calG$ has a source node $\sigma \in \calV$ and a
  destination $\tau \in \calV$ Each node $\alpha \in \calV$ represent
  a VNF that has a processing demand $p(\alpha)$.  Without loss of
  generality, we assume that $p(\sigma) = p(\tau) = 0$.  Each edge
  $\eps \in \calE$ has a bandwidth demand $b(\eps)$.
  
\item[Placement costs:]
%
  There is a non-negative cost $c(\alpha,v)$ for placing the VNF
  $\alpha$ in $v$.  We assume that $c(\sigma,s) = 0$ and $c(\sigma,v)
  = \infty$, for every $v \neq s$.  Similarly, we assume that
  $c(\tau,t) = 0$ and $c(\tau,v) = \infty$, for every $v \neq t$.
\end{description}

A solution consists of the following:
\begin{description}
\item[Virtual path:] A path from $\sigma$ to $\tau$ in $\calG$, namely
  a sequence of vertices $\alpha_0,\ldots,\alpha_\ell$, where
  $\alpha_0 = \sigma$, $\alpha_\ell = \tau$, and
  $(\alpha_j,\alpha_{j+1}) \in \calE$, for every $j \in
  \set{0,\ldots,\ell-1}$.

\item[Physical path:] A simple path from $s$ to $t$ in $G$, namely a
  sequence of node $v_0,\ldots,v_k$, where $v_0 = s$, $v_k = t$, and
  $(v_i,v_{i+1}) \in E$, for every $i \in \set{0,\ldots,k-1}$.
  
\item[Placement:] A function $f$ that maps the virtual path to the
  physical path.  Formally, a placement is a function $f:
  \set{\alpha_0,\ldots,\alpha_\ell} \to \set{v_0,\ldots,v_k}$
  such that
\begin{enumerate}
\item If $f(\alpha_j) = v_i$ and $f(\alpha_{j+1}) = v_{i'}$, then $i
  \leq i'$.
\item $b(\alpha_j,\alpha_{j+1}) \leq b(v_i,v_{i+1})$, for every
  $(v_i,v_{i+1}) \in \bar{f}(\alpha_j,\alpha_{j+1})$, where
  $\bar{f}(\alpha_j,\alpha_{j+1})$ to be the set of physical edges
  that correspond to it, i.e.,
\[
\bar{f}(\alpha_j,\alpha_{j+1}) \eqdf \set{(v_i,v_{i+1}) : i' \leq i < i''}
~,
\]
  where $f(\alpha_j) = v_{i'}$ and $f(\alpha_{j+1}) = v_{i''}$.

\item $\sum_{\alpha \in f^{-1}(v)} p(\alpha) \leq p(v)$, where
  $f^{-1}(v) \eqdf \set{\alpha : f(\alpha) = v}$.
 
\end{enumerate}
\end{description}

An example of an \scp instance and a solution is given in
Figure~\ref{fig:solution}.

\begin{figure}
\centering
\scalebox{1}{
\begin{tikzpicture}[every node/.style={default node}, very thick]

\begin{scope}[random seed=1, spring layout, node distance=1.8cm]
\node(a) {a};
\node(b) {b};
\node(c) {c};
\node(d) {d};
\node(e) {e};
\node(f) {f};
\node(g) {g};
\node(h) {h};
\node(i) {i};
\node(j) {j};

\graph{
	(a) -> (b);
	(b) -> {(d), (e)};
	(d) -> {(f), (h)};
	(f) -> {(g), (h), (c)};
	(h) -> {(g), (i), (e)};
	{[edges={blue, dashed}]
		(a) -> (c) -> (d) -> (e) -> (i) -> (g) -> (j);
	}
};

\node(pa) {a'};
\node(pb) {b'};
\node(pc) {c'};
\node(pd) {d'};
\node(pe) {e'};
\node(pf) {f'};
\node(pg) {g'};
\node(ph) {h'};

\graph{
	(pa) -> (pb);
	(pb) -> {(pd), (pe)};
	(pd) -> {(pf), (ph)};
	(pf) -> {(pg), (ph), (pc)};
	(ph) -> {(pg), (pe)};
	(pe) -> (ph);
};
\end{scope}

\begin{scope}[dotted, orange, -]
\draw
(a.north east) to[out=135, in=45] node(j1)[midway, clear node] {}
(c.north west) to[out=225, in=135]
(d.south west) to[out=-45, in=-45]
(a.north east)
;

\draw
(e.east) to[out=-90, in=-90] node(j2)[midway, clear node] {}
(i.west) to[out=90, in=90]
(e.east)
;

\draw
(g.east) to[out=-90, in=-90] node(j3)[midway, clear node] {}
(j.west) to[out=90, in=90]
(g.east)
;

\draw[->] (j1) to[bend left] (pa);
\draw[->] (j2) to[bend right] (pe);
\draw[->] (j3) to[bend right] (pg);
\end{scope}
	

\node[clear node] at(-5,0) {Virtual Graph};
\node[clear node] at(5, -2){Physical Graph};

\end{tikzpicture}
}
\caption{???}
\label{fig:solution}
\end{figure}

The cost of a placement $f$ is:
\[
\sum_j c(\alpha_j,f(\alpha_j))
~.
\]
In \scp the goal is to find a feasible placement of a virtual path
into the physical DAG that minimizes the cost.

%%%%%

We also consider an extended version of \scp in which each physical
link causes a delay and there is an upper bound $T$ on the total
delay.  More formally, each pair of a virtual edge $\eps$ and physical
edge $e$ is associated with a latency $t(\eps,e)$.
%
Given a placement $f: \set{\alpha_0,\ldots,\alpha_\ell} \to
\set{v_0,\ldots,v_k}$, the total latency of the solution is given by
\[
T(f) = \sum_j \sum_{e \in \bar{f}(\eps)} t(\eps,e)
~.
\]
In this case there is an additional constraint that $T(f) \leq T$.


%%%%%

\paragraph*{\bf Definitions and notation.}
%
Define $n_p \eqdf \abs{V}$ and $m_p \eqdf \abs{E}$.  Similarly, define
$n_v \eqdf \abs{\calV}$ and $m_v \eqdf \abs{\calE}$.

Given a virtual DAG $\calG$, we assume the existence of a topological
sorting of the vertices and write $\alpha \prec_v \beta$ if $\alpha$
precedes $\beta$ in this ordering.  If the physical network is a DAG,
then we make a similar assumption and write $v \prec_p u$ if $v$
precedes $u$ in this ordering.




%%%%%%%%%%%%%%%%%%%%%%%%%%%%%%%%%%%%%%%%%%%%%%%%%%%%%%%%%%%%%%%%%%%%%

\section{Hardness}

In this section we present two hardness results.  First, we show that
even the feasibility question of \scp is NP-hard, and therefore no
approximation algorithm exists for \scp.  In addition, we show that
\scp is NP-hard even if $\abs{V \setminus \set{s,t}} = 1$.  We also
show that \scp is NP-hard even if both the physical network and the
virtual DAG are simple paths.

We show that finding a feasible solution is NP-hard.

\begin{theorem}
Feasibility of \scp is NP-hard
\end{theorem}
\begin{proof}
We use a reduction from \textsc{Hamiltonian Path} that is known to be
NP-hard~\cite{GarJoh79}.
%
Given an instance $H$ of \textsc{Hamiltonian Path} we construct an
instance of \scp.  For the physical network we have that $G$, where
$V(G) = V(H) \cup \set{s,t}$, and $E(G) = E(H) \cup \set{(s,v),(v,t) :
  v \in V(H)}$.  In addition, $p(v) = 1$, for every $v$, and $b(e) =
1$, for every $e$.  The virtual DAG $\calG$ is a path containing $n+2$
virtual functions, $\sigma = \alpha_0, \alpha_1, \ldots, \alpha_n,
\alpha_{n+1} = \tau$, where $p(\alpha_i) = 1$, for every $i \in
\set{1,\ldots,n}$.  Also, $b(\alpha_i,\alpha_{i+1}) = 1$, for every
$i$.

The construction can clearly be computed in polynomial time.
%
An Hamiltonian Path in $G$, induces a \scp solution that follows the
path, i.e., $\alpha_i$ is placed in the $i$th vertex along the path.
%
One the other hand, since all demands and capacities are $1$, no two
VNFs can share a physical node.  Hence a \scp solution induces an
Hamiltonian path.
%
\qed
\end{proof}

Next, we show that \scp is NP-hard even if the physical network is a
DAG.

\begin{theorem}
\scp is NP-hard, even if $\abs{V \setminus \set{s,t}} = 1$.
\end{theorem}
\begin{proof}
The prove the theorem using a reduction from \textsc{Partition}.
Given a \textsc{Partition} instance $\set{a_1, \ldots, a_n}$ we
construct an \scp instance as follows.
%
The physical network $G$ contains three nodes: $s$, $v$, and $t$, and
there are two edges $(s,v)$ and $(v,t)$.  The capacity of $v$ is $p(v)
= \half \sum_i a_i$.  Edge capacities are zero.
%
The virtual DAG is composed of $2n+2$ vertices, namely
\[
\calV
= \set{\sigma,\tau}
  \cup \set{\alpha_1, \ldots, \alpha_n}
  \cup \set{\beta_1, \ldots, \beta_n}
~.
\]
Also,
\[
\calE
= \bigcup_i \set{ \set{\alpha_i,\beta_i} \times \set{\alpha_{i+1},\beta_{i+1}} } 
  \cup
  \set{ (\sigma,\alpha_1), (\sigma,\beta_1), (\alpha_n,t), (\beta_n,t) }
~.
\]
The DAG is shown in Figure~\ref{fig:dag}.
%
The demands are $p(\alpha_i) = a_i$ and $p(\beta_i) = 0$, for every
$i$.  Also, $b(\eps) = 0$, for every $\eps \in \calE$.  The costs are:
$c(\alpha_i,v) = 0$ and $c(\beta_i,v) = a_i$, for every $i$.

The \scp instance can be computed in polynomial time.
%
Also, it is not hard to verify that $a \in \textsc{Partition}$ if and
only if there exists a solution whose cost is $\half \sum_i a_i$.
%
\qed
\end{proof}

\begin{figure}
\centering
\begin{tikzpicture}[every node/.style={default node}]

\begin{scope}[layered layout, grow=right, sibling distance=2cm, level distance=2cm]

\node(v1p) {$\beta_1$};
\node(v1m) {$\alpha_1$};
\node(v2p) {$\beta_2$};
\node(v2m) {$\alpha_2$};
\node[clear node, inner sep=5mm](dotsp) {$\ldots$};
\node[clear node, inner sep=5mm](dotsm) {$\ldots$};
\node(vnp) {$\beta_n$};
\node(vnm) {$\alpha_n$};

\graph{
	s -> {(v1p), (v1m)};
	(v1p) -> {(v2p), (v2m)};
	(v1m) -> {(v2p), (v2m)};
	(v2p) -> {(dotsp), (dotsm)};
	(v2m) -> {(dotsp), (dotsm)};
	(dotsp) -> {(vnp), (vnm)};
	(dotsm) -> {(vnp), (vnm)};
	{(vnp), (vnm)} -> t;
};

\end{scope}

\end{tikzpicture}

\caption{???}
\label{fig:dag}
\end{figure}

\dror{Figure 2: source and destination should be fixed}

\begin{theorem}
\scp is NP-hard, even if both the physical network and the virtual DAG
are paths.
\end{theorem}
\begin{proof}

\dror{reduction from partition using latency}
  
%
\qed
\end{proof}

%%%%%%%%%%%%%%%%%%%%%%%%%%%%%%%%%%%%%%%%%%%%%%%%%%%%%%%%%%%%%%%%%%%%%

\section{Algorithm}


%%%%%%%%%%%%%%%%%%%%%%%%%%%%%%%%%%%%%%%%%%%%%%%%%%%%%%%%%%%%%%%%%%%%%

\section{Latency}


%%%%%%%%%%%%%%%%%%%%%%%%%%%%%%%%%%%%%%%%%%%%%%%%%%%%%%%%%%%%%%%%%%%%%

\bibliographystyle{abbrv}
\bibliography{chain}

\end{document}
