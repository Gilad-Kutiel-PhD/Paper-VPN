
Computer communication networks are in constant need of expansion to
cope with the ever growing traffic.  As networks grow, management and
maintenance become more and more complicated.
%
Current developments aim to improve the utilization of network
resources include the detachment of network applications from network
infrastructure and the transition from network planning to network
programming.

One aspect of network programming is to manage resources from a
central point of view, namely to make decisions based on availability,
network status, required quality of service, and the identity of the
client.  Hence, a central issue is a central agent that is able to
received reports from network components and client requests, and as a
result can alter the allocation of resources in the networks.  This
approach is called Software Defined Networking (SDN), where there is a
separation between the physical layer and the routing and management
layers.

A complimentary approach is Network Function Virtualization (NFV).
Instead of relying on special purpose machines, network applications
become virtual network functions (VNF) that are executed on generic
machines and can be placed in various locations in the network.
Virtualization increases the flexibility of resource allocation and
thus the utilization of the network resources.
%
Internet Service Providers (ISPs) that provide services to clients
benefit from NFV, since it helps to better utilize their physical
network.  In addition, when network services are virtualized, an ISP
may support \emph{service chaining}, namely a compound service that
consists of a sequence of VNFs.  Furthermore, one may offer a compound
service a service that may be obtained using one of several sequences
of VNFs.

In this paper we study the allocation problem of a compund request for
a service chain in a software-defined network that supports NFV.
Given a (physical) network that contains servers with limited
processing power and of links with limited bandwidth, a service chain
is a sequence of virtual network functions (VNFs) that service a
certain flow request in the network.
%
The allocation of a service chain consists of routing and VNF
placement.  That is, each VNF from the seqeunce is placed in a server
along a path, and it is feasible if each server can handle the VNFs
that are assigned to it, and if each link on the path can carry the
flow that is assigned to it.
%
A request for service is composed of a source and a destination in the
network, an upper bound on the total latency, and a specification of
all service chains that are considered valid for this request.
%
Each pair of server and VNF are associated with a cost for placing the
network function in the server.  This cost measures compatibility of a
VNF to a server (infinite costs means ``incompatible'').  Given a
request, the goal is to find a valid service chain of minimum total
cost or to identify that a valid service chain does not exist.

????? References!!! ?????

?????  Explain one shot approach ?????

\paragraph*{\bf Related work.}
%



\paragraph*{\bf Our results.}
%
We show that even the feasibility problem is NP-hard in general
di-graphs.  Hence we first focus on directed acyclic graphs (DAGs).
We show that the problem is still NP-hard in DAGs and present an
FPTAS.
%
Based on out PTAS, we also provide a randomized algorithm for
instances in which the service chain passes through a bounded number
of servers whose degree is larger than two.

?????

\bigskip




